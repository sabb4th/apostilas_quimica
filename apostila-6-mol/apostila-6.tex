% LISTA DE EXERCÍCIOS Template using "book"
% Created by Milena Lima 
%Email:milenascimentolima@gmail.com
% Science Project at school 2017-FAPEAM
% View: https://www.overleaf.com/read/syjxcffdygch
%=======================================================
%------------LISTA DE EXERCÍCIOS
%=======================================================
\documentclass[12pt,a4paper,oneside,openany]{book} 
%=======================================================
%-------------PACOTES 
%=======================================================
\usepackage[top=1cm,left=1cm,right=1.5cm,bottom=2cm]{geometry}
\usepackage[T1]{fontenc}%Especif. a codif. de caracteres
\usepackage{ae}%Auxílio para fontes e pdf
\usepackage[utf8]{inputenc}
\usepackage{lipsum}%Gerar Texto Aleatório
\usepackage[brazil]{babel}%Traduzir para Português
\usepackage{indentfirst}%Faz indentações em parágrafos
\usepackage{graphicx}%Permite incluir figuras
\usepackage{subfig} %para criar sub figuras
\usepackage{float}% figuras
\usepackage{tabularx}
\usepackage{ragged2e}
\usepackage{multirow}
\usepackage[dvipsnames]{xcolor}%Admitir cores
\usepackage{amsmath,amssymb,amsthm}%Incluir expressões  matemáticas, equações, teoremas, símbolos, etc
\usepackage{lastpage}%Incluir Ficha catalográfica
\usepackage{epigraph}%Incluir Epígrafo
\usepackage{enumerate}
\usepackage{enumitem}
\newlist{questions}{enumerate}{3}
\setlist[questions]{label=\arabic*.}
\newcommand{\question}{\item}
\setlist[enumerate,1]{% (
leftmargin=*, itemsep=12pt, label={\textbf{\arabic*.)}}}
%---
\newlist{partes}{enumerate}{3}
\setlist[partes]{label=(\alph*)}
\newcommand{\parte}{\item}
%---
\newlist{subpartes}{enumerate}{3}
\setlist[subpartes]{label=\roman*)}
\newcommand{\subparte}{\item}
%---
\usepackage{array}
\usepackage{tikz}
\newcommand*\circled[1]{\tikz[baseline=(char.base)]{\node[shape=circle,draw,inner sep=2pt] (char) {#1};}}
\usepackage[sort&compress,round,comma,authoryear]{natbib}
\usepackage{makeidx}
\usepackage[colorlinks=true,urlcolor=magenta,citecolor=red,linkcolor=violet,bookmarks=true]{hyperref}
\usepackage{lscape}%Altera a orientação de uma página
\usepackage{pdflscape}
\usepackage{epstopdf} %converte figs eps em figs pdf
\usepackage{booktabs}
\usepackage{pdfpages}
\usepackage{textcomp}
\usepackage[many]{tcolorbox}
\usepackage{empheq}
\usepackage{tasks}%lista alfabética
\pagestyle{plain}
\usepackage{mhchem}
%================================================================
%------------DIGITE AQUI
%===============================================================
\newcommand{\orgao}{Universidade do Estado do Pará}
\newcommand{\instituto}{Movimento de Educação Popular e Inclusiva do Jurunas}
\newcommand{\departamento}{EEFM Arthur Porto}
\newcommand{\curso}{Ciências da Computação}
\newcommand{\professor}{Elias Sabát}
\newcommand{\disciplina}{Química}
\newcommand{\titulo}{6\textsuperscript{a} Lista de Exercícios: Mol}
\newcommand{\data}{\today}
\newcommand{\aluno}{ALUNO:}
\newcommand{\email}{{\bf Professor: Isaias Tobelem}}
\newcommand{\turma}{kkkkkkkkkkkkkkkkkkkkkkkkkk}
%===========================================================
\newcommand{\X}{\mathbf{X}}
\newcommand{\x}{\mathbf{x}}
\newcommand{\Z}{\mathbf{Z}}
\newcommand{\z}{\mathbf{z}}
\newcommand{\y}{\boldsymbol{y}}
\newcommand{\balpha}{\mbox{${ \bm \alpha}$}}
\newcommand{\bmu}{\mbox{${\bm \mu}$}}
\newcommand{\bbeta}{\mbox{${\bm \beta}$}}
\newcommand{\bteta}{\mbox{${\bm \theta}$}}
\newcommand{\bgama}{\mbox{${\bm \gamma}$}}
\newcommand{\bxi}{\mbox{${\bm \xl}$}}
\newcommand{\bvarphi}{\mbox{${ \bm \varphi}$}}
\newcommand{\SZ}{\mbox{ $Z$}}
\newcommand{\muz}{\mu_{z,l}}
\newcommand{\muo}{\mu_{0,l}}
\newcommand{\etao}{\eta_{0,l}}
\newcommand{\etaz}{\eta_{z,l}}
\newcommand{\xbeta}{x_{l}\bgama}
\newcommand{\mui}{\mu_{l}}
\newcommand{\zetaind}{\zeta \mathtt{I}_{\{s_{l} \in Z \}}} 
\newcommand{\spz}{ s_{l} \in z}
\newcommand{\snpz}{ s_{l} \notin z}
\newcommand{\sps}{ s_{l} \in S}
\newcommand{\gphi}{ \Gamma(\phi)}
\newcommand{\scan}{ \Lambda_{z}}
\newcommand{\gmuop}{ \Gamma(\muo\phi)}
\newcommand{\gmuzp}{ \Gamma(\muz \phi)}
\newcommand{\dlobeta}{ \frac{\parteial l_{0}(\bgama, \phi, 0) }{\parteial \bgama}}
\newcommand{\lz}{  l_{z}(\bgama, \phi, \tau)}
\newcommand{\lo}{  l_{0}(\beta, \phi, 0)}
\newcommand{\E}{\mathbb{E}}
\newcommand{\dis}{\displaystyle}
\linespread{1.0}%espaço entre linhas
\begin{document}
%%%%%%%%%%%%%%%%%%%%%%%%%%%%%%%%%%%%%%%%%%%%%%%%%%%%%%%%
%                      CABEÇALHO                     %
%%%%%%%%%%%%%%%%%%%%%%%%%%%%%%%%%%%%%%%%%%%%%%%%%%%%%%%%
\begin{table}[H]
\centering
\begin{tabular*}{\textwidth}{l@{\extracolsep{\fill}}l@{\extracolsep{\fill}}}
\begin{tabular}[l]{@{}l@{}}\textbf{\orgao}\\\textbf{\instituto}\\\textbf{\departamento} \end{tabular} & \begin{tabular}[l]{@{}l@{}}\textbf{Professor: \professor}\\ {\email}\\ \textbf{Disciplina: \disciplina}\end{tabular}                                                       
\end{tabular*}
\end{table}
\begin{center}
\rule[2ex]{\textwidth}{1pt}\\
{\Large{\titulo}}
\end{center}
\rule[2ex]{\textwidth}{1pt}\\

\textbf{Assuntos:} Unidade de Massa Atômica, Massa Molecular e Massa Molar 

\begin{questions}[label=\protect\circled{\bfseries\arabic*}]
%%%%%%%%%%%%%%%%%%%%%%%%%%%%%%%%%%%%%%%%%%%%%%%%%%%%%%%%
%                      Questões                   %
%%%%%%%%%%%%%%%%%%%%%%%%%%%%%%%%%%%%%%%%%%%%%%%%%%%%%%%%

%=========================================================
\question Calcule as massas moleculares das seguintes substâncias:

a) $C_{2}H_{6}$

b) $SO_2$

c) $CaCO_{3}$

d) $NaHSO_4$

e) $CH_{3}COONa$

\question Determine o número de átomos contidos em 100,0 g de álcool etílico ($C_{2}H_{6}O$).

\question Em uma amostra de 4,3 g de hexano ($C_{6}H_{14}$) encontramos quantos mol e quantas moléculas?

\question A quantos gramas correspondem $3 \cdot 10^{24}$ átomos de alumínio?

\question Qual é a massa correspondente a 5 mols de alumínio? (Massa atômica do alumínio = 27)

\question  Três mols de benzeno ($C_{6}H_{6}$) contêm uma massa em g de:

\question $1,8 \cdot 10^{23}$ moléculas de uma substância A têm massa igual a 18,0 g. A massa molar de A, em g/mol, vale:

\question Silicatos são compostos de grande importância nas indústrias de cimento, cerâmica e vidro. Quantos
gramas de silício há em 2,0 mols do silicato natural $Mg_{2}SiO_{4}$?



\question Quantas moléculas existem em 88 g de dióxido de carbono ($CO_{2}$)? (Massas atômicas: C = 12; O = 16; constante de
Avogadro = $6,02 \cdot 10^{23}$)

\begin{partes}

\parte $6,02 \cdot 10^{23}$
\parte $1,2 \cdot 10^{24}$
\parte 2 mols
\parte 1 mol
\parte $12,04 \cdot 10^{23}$

\end{partes}

\question Submetida a um tratamento médico, uma pessoa ingeriu um comprimido contendo 45 mg de ácido acetilsalicílico
($C_{9}H_{8}O_{4}$). Considerando a massa molar do $C_{9}H_{8}O_{4}$ 180 g/mol, e o número de Avogadro $6,02 \cdot 10^{23}$, é correto afirmar que o número de moléculas da substância ingerida é:

\begin{partes}

\parte $1,5 \cdot 10^{20}$
\parte $2,4 \cdot 10^{23}$
\parte $3,4 \cdot 10^{23}$
\parte $4,5 \cdot 10^{20}$
\parte $6 \cdot 10^{23}$

\end{partes}

\question Um recipiente contém 2,0 mols de cloro gasoso. O número de moléculas do gás é:

\begin{partes}

\parte $2,4 \cdot 10^{23}$ 
 \parte 4,0
 \parte $1,2 \cdot 10^{24}$ 
 \parte 2,0
 \parte $1,2 \cdot 10^{23}$

\end{partes}

\question A quantidade de átomos em um mol de
ácido sulfúrico é:

\begin{partes}

\parte $3 \cdot 6,02 \cdot 10^{23}$ átomos/mol
\parte $4 \cdot 6,02 \cdot 10^{23}$ átomos/mol
\parte $5 \cdot 6,02 \cdot 10^{23}$ átomos/mol
\parte $6 \cdot 6,02 \cdot 10^{23}$ átomos/mol
\parte $7 \cdot 6,02 \cdot 10^{23}$ átomos/mol

\end{partes}

\question \citep{ENEM(2013)}
brasileiro consome em média 500 miligramas de cálcio por dia, quando a quantidade
recomendada é o dobro. Uma alimentação balanceada é a melhor decisão para evitar problemas no
futuro, como a osteoporose, uma doença que atinge os ossos. Ela se caracteriza pela diminuição
substancial de massa óssea, tornando os ossos frágeis e mais suscetíveis a fraturas.

Considerando-se o valor de $6 \cdot 10^{23}$  para a constante de Avogadro e a massa molar do cálcio igual a 40 g/mol, qual a quantidade mínima diária de átomos de cálcio a ser ingerida para que
uma pessoa supra suas necessidades?

\begin{partes}

\parte $7,5 \cdot 10^{21}$
\parte $1,5 \cdot 10^{22}$
\parte $7,5 \cdot 10^{23}$
\parte $1,5 \cdot 10^{25}$
\parte $4,8 \cdot 10^{25}$

\end{partes}

\question \citep{ENEM(2018)}
As soluções de hipoclorito de sódio têm ampla aplicação como desinfetantes e alvejantes. Em uma empresa de limpeza, o responsável pela área de compras deve decidir entre dois fornecedores que têm produtos similares, mas com diferentes teores de cloro. 

Um dos fornecedores vende baldes de 10 kg de produto granulado, contendo $65\%$ de cloro
ativo, a um custo de 65,00. Outro fornecedor oferece, a um custo de 20,00, bombonas de 50 kg de produto líquido contendo $10\%$ de cloro ativo.

Considerando apenas o quesito preço por kg de cloro ativo e desprezando outras variáveis, para cada bombona de 50 kg haverá uma economia de

\begin{partes}

\parte 4,00.
\parte 6,00.
\parte 10,00.
\parte 30,00.
\parte 45,00.

\end{partes}


\question \citep{ENEM(2012)}
No Japão, um movimento nacional para a promoção da luta contra o aquecimento global leva o slogan: 1 pessoa, 1 dia, 1 kg de $CO_{2}$ a menos! A ideia é cada pessoa reduzir em 1 kg a quantidade
de $CO_{2}$ emitida todo dia, por meio de pequenos gestos ecológicos, como diminuir a queima de gás
de cozinha.

Considerando um processo de combustão completa de um gás de cozinha composto
exclusivamente por butano ($C_{4}H_{10}$), a mínima quantidade desse gás que um japonês deve deixar de queimar para atender à meta diária, apenas com esse gesto, é de

Dados: $CO_{2}$ (44 g/mol); $C_{4}H_{10}$ (58 g/mol)

\begin{partes}

\parte 0,25 kg.
\parte 0,33 kg.
\parte 1,0 kg.
\parte 1,3 kg.
\parte 3,0 kg. 

\end{partes}



%================================================================
\end{questions}
%===========================================================
%          BIBLIOGRAFIA
%===========================================================
\begin{thebibliography}{99}
\thispagestyle{empty}%myheadings
%===========================================================
%==============Livro
\bibitem[ENEM(2013)]{ENEM(2013)}
[1] \textbf{ENEM 2013}
(Exame Nacional do Ensino Médio).
\emph{INEP-Instituto Nacional de Estudos e Pesquisas Educacionais Anísio Teixeira}.{Ministério da Educação}. 

\bibitem[ENEM(2009)]{ENEM(2009)}
[2] \textbf{ENEM 2009}
(Exame Nacional do Ensino Médio).
\emph{INEP-Instituto Nacional de Estudos e Pesquisas Educacionais Anísio Teixeira}.{Ministério da Educação}. 

\bibitem[ENEM(2011)]{ENEM(2011)}
[3] \textbf{ENEM 2011}
(Exame Nacional do Ensino Médio).
\emph{INEP-Instituto Nacional de Estudos e Pesquisas Educacionais Anísio Teixeira}.{Ministério da Educação}. 

\bibitem[ENEM(2010)]{ENEM(2010)}
[4] \textbf{ENEM 2010}
(Exame Nacional do Ensino Médio).
\emph{INEP-Instituto Nacional de Estudos e Pesquisas Educacionais Anísio Teixeira}.{Ministério da Educação}. 

\bibitem[ENEM(2025)]{ENEM(2025)}
[5] \textbf{ENEM 2025}
(Exame Nacional do Ensino Médio).
\emph{INEP-Instituto Nacional de Estudos e Pesquisas Educacionais Anísio Teixeira}.{Ministério da Educação}.

\bibitem[ENEM(2024)]{ENEM(2024)}
[6] \textbf{ENEM 2024}
(Exame Nacional do Ensino Médio).
\emph{INEP-Instituto Nacional de Estudos e Pesquisas Educacionais Anísio Teixeira}.{Ministério da Educação}.

\bibitem[ENEM(2022)]{ENEM(2022)}
[7] \textbf{ENEM 2022}
(Exame Nacional do Ensino Médio).
\emph{INEP-Instituto Nacional de Estudos e Pesquisas Educacionais Anísio Teixeira}.{Ministério da Educação}.

\bibitem[ENEM(2020)]{ENEM(2020)}
[8] \textbf{ENEM 2020}
(Exame Nacional do Ensino Médio).
\emph{INEP-Instituto Nacional de Estudos e Pesquisas Educacionais Anísio Teixeira}.{Ministério da Educação}.

\bibitem[ENEM(2017)]{ENEM(2017)}
[9] \textbf{ENEM 2017}
(Exame Nacional do Ensino Médio).
\emph{INEP-Instituto Nacional de Estudos e Pesquisas Educacionais Anísio Teixeira}.{Ministério da Educação}.

\bibitem[ENEM(2012)]{ENEM(2012)}
[10] \textbf{ENEM 2012}
(Exame Nacional do Ensino Médio).
\emph{INEP-Instituto Nacional de Estudos e Pesquisas Educacionais Anísio Teixeira}.{Ministério da Educação}.

\bibitem[ENEM(2018)]{ENEM(2018)}
[1] \textbf{ENEM 2018}
(Exame Nacional do Ensino Médio).
\emph{INEP-Instituto Nacional de Estudos e Pesquisas Educacionais Anísio Teixeira}.{Ministério da Educação}. 

\bibitem[ENEM(2009)]{ENEM(2009)}
[2] \textbf{ENEM 2009}
(Exame Nacional do Ensino Médio).
\emph{INEP-Instituto Nacional de Estudos e Pesquisas Educacionais Anísio Teixeira}.{Ministério da Educação}. 

\bibitem[ENEM(2011)]{ENEM(2011)}
[3] \textbf{ENEM 2011}
(Exame Nacional do Ensino Médio).
\emph{INEP-Instituto Nacional de Estudos e Pesquisas Educacionais Anísio Teixeira}.{Ministério da Educação}. 

\bibitem[ENEM(2010)]{ENEM(2010)}
[4] \textbf{ENEM 2010}
(Exame Nacional do Ensino Médio).
\emph{INEP-Instituto Nacional de Estudos e Pesquisas Educacionais Anísio Teixeira}.{Ministério da Educação}. 

\bibitem[ENEM(2025)]{ENEM(2025)}
[5] \textbf{ENEM 2025}
(Exame Nacional do Ensino Médio).
\emph{INEP-Instituto Nacional de Estudos e Pesquisas Educacionais Anísio Teixeira}.{Ministério da Educação}.

\bibitem[ENEM(2024)]{ENEM(2024)}
[6] \textbf{ENEM 2024}
(Exame Nacional do Ensino Médio).
\emph{INEP-Instituto Nacional de Estudos e Pesquisas Educacionais Anísio Teixeira}.{Ministério da Educação}.

\bibitem[ENEM(2022)]{ENEM(2022)}
[7] \textbf{ENEM 2022}
(Exame Nacional do Ensino Médio).
\emph{INEP-Instituto Nacional de Estudos e Pesquisas Educacionais Anísio Teixeira}.{Ministério da Educação}.

\bibitem[ENEM(2020)]{ENEM(2020)}
[8] \textbf{ENEM 2020}
(Exame Nacional do Ensino Médio).
\emph{INEP-Instituto Nacional de Estudos e Pesquisas Educacionais Anísio Teixeira}.{Ministério da Educação}.

\bibitem[ENEM(2017)]{ENEM(2017)}
[9] \textbf{ENEM 2017}
(Exame Nacional do Ensino Médio).
\emph{INEP-Instituto Nacional de Estudos e Pesquisas Educacionais Anísio Teixeira}.{Ministério da Educação}.

\bibitem[ENEM(2012)]{ENEM(2012)}
[10] \textbf{ENEM 2012}
(Exame Nacional do Ensino Médio).
\emph{INEP-Instituto Nacional de Estudos e Pesquisas Educacionais Anísio Teixeira}.{Ministério da Educação}.

\bibitem[ENEM(2014)]{ENEM(2014)}
[11] \textbf{ENEM 2014}
(Exame Nacional do Ensino Médio).
\emph{INEP-Instituto Nacional de Estudos e Pesquisas Educacionais Anísio Teixeira}.{Ministério da Educação}.

\bibitem[ENEM(2015)]{ENEM(2015)}
[11] \textbf{ENEM 2015}
(Exame Nacional do Ensino Médio).
\emph{INEP-Instituto Nacional de Estudos e Pesquisas Educacionais Anísio Teixeira}.{Ministério da Educação}.

%===========================================================
\end{thebibliography}
\end{document}
