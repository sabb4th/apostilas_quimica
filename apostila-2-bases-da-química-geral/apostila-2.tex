% LISTA DE EXERCÍCIOS Template using "book"
% Created by Milena Lima 
%Email:milenascimentolima@gmail.com
% Science Project at school 2017-FAPEAM
% View: https://www.overleaf.com/read/syjxcffdygch
%=======================================================
%------------LISTA DE EXERCÍCIOS
%=======================================================
\documentclass[12pt,a4paper,oneside,openany]{book} 
%=======================================================
%-------------PACOTES 
%=======================================================
\usepackage[top=1cm,left=1cm,right=1.5cm,bottom=2cm]{geometry}
\usepackage[T1]{fontenc}%Especif. a codif. de caracteres
\usepackage{ae}%Auxílio para fontes e pdf
\usepackage[utf8]{inputenc}
\usepackage{lipsum}%Gerar Texto Aleatório
\usepackage[brazil]{babel}%Traduzir para Português
\usepackage{indentfirst}%Faz indentações em parágrafos
\usepackage{graphicx}%Permite incluir figuras
\usepackage{subfig} %para criar sub figuras
\usepackage{float}% figuras
\usepackage{tabularx}
\usepackage{ragged2e}
\usepackage{multirow}
\usepackage[dvipsnames]{xcolor}%Admitir cores
\usepackage{amsmath,amssymb,amsthm}%Incluir expressões  matemáticas, equações, teoremas, símbolos, etc
\usepackage{lastpage}%Incluir Ficha catalográfica
\usepackage{epigraph}%Incluir Epígrafo
\usepackage{enumerate}
\usepackage{enumitem}
\newlist{questions}{enumerate}{3}
\setlist[questions]{label=\arabic*.}
\newcommand{\question}{\item}
\setlist[enumerate,1]{% (
leftmargin=*, itemsep=12pt, label={\textbf{\arabic*.)}}}
%---
\newlist{partes}{enumerate}{3}
\setlist[partes]{label=(\alph*)}
\newcommand{\parte}{\item}
%---
\newlist{subpartes}{enumerate}{3}
\setlist[subpartes]{label=\roman*)}
\newcommand{\subparte}{\item}
%---
\usepackage{array}
\usepackage{tikz}
\newcommand*\circled[1]{\tikz[baseline=(char.base)]{\node[shape=circle,draw,inner sep=2pt] (char) {#1};}}
\usepackage[sort&compress,round,comma,authoryear]{natbib}
\usepackage{makeidx}
\usepackage[colorlinks=true,urlcolor=magenta,citecolor=red,linkcolor=violet,bookmarks=true]{hyperref}
\usepackage{lscape}%Altera a orientação de uma página
\usepackage{pdflscape}
\usepackage{epstopdf} %converte figs eps em figs pdf
\usepackage{booktabs}
\usepackage{pdfpages}
\usepackage{textcomp}
\usepackage[many]{tcolorbox}
\usepackage{empheq}
\usepackage{tasks}%lista alfabética
\pagestyle{plain}
\usepackage{mhchem}
%================================================================
%------------DIGITE AQUI
%===============================================================
\newcommand{\orgao}{Universidade do Estado do Pará}
\newcommand{\instituto}{Movimento de Educação Popular e Inclusiva do Jurunas}
\newcommand{\departamento}{EEFM Arthur Porto}
\newcommand{\curso}{Ciências da Computação}
\newcommand{\professor}{Elias Sabát}
\newcommand{\disciplina}{Química}
\newcommand{\titulo}{2\textsuperscript{a} Lista de Exercícios: Bases da Química Geral}
\newcommand{\data}{\today}
\newcommand{\aluno}{ALUNO:}
\newcommand{\email}{{\bf Professor: Isaias Tobelem}}
\newcommand{\turma}{kkkkkkkkkkkkkkkkkkkkkkkkkk}
%===========================================================
\newcommand{\X}{\mathbf{X}}
\newcommand{\x}{\mathbf{x}}
\newcommand{\Z}{\mathbf{Z}}
\newcommand{\z}{\mathbf{z}}
\newcommand{\y}{\boldsymbol{y}}
\newcommand{\balpha}{\mbox{${ \bm \alpha}$}}
\newcommand{\bmu}{\mbox{${\bm \mu}$}}
\newcommand{\bbeta}{\mbox{${\bm \beta}$}}
\newcommand{\bteta}{\mbox{${\bm \theta}$}}
\newcommand{\bgama}{\mbox{${\bm \gamma}$}}
\newcommand{\bxi}{\mbox{${\bm \xl}$}}
\newcommand{\bvarphi}{\mbox{${ \bm \varphi}$}}
\newcommand{\SZ}{\mbox{ $Z$}}
\newcommand{\muz}{\mu_{z,l}}
\newcommand{\muo}{\mu_{0,l}}
\newcommand{\etao}{\eta_{0,l}}
\newcommand{\etaz}{\eta_{z,l}}
\newcommand{\xbeta}{x_{l}\bgama}
\newcommand{\mui}{\mu_{l}}
\newcommand{\zetaind}{\zeta \mathtt{I}_{\{s_{l} \in Z \}}} 
\newcommand{\spz}{ s_{l} \in z}
\newcommand{\snpz}{ s_{l} \notin z}
\newcommand{\sps}{ s_{l} \in S}
\newcommand{\gphi}{ \Gamma(\phi)}
\newcommand{\scan}{ \Lambda_{z}}
\newcommand{\gmuop}{ \Gamma(\muo\phi)}
\newcommand{\gmuzp}{ \Gamma(\muz \phi)}
\newcommand{\gumuop}{ \Gamma((1-\muo)\phi)}
\newcommand{\gumuzp}{ \Gamma((1-\muz)\phi)}
\newcommand{\dlobeta}{ \frac{\parteial l_{0}(\bgama, \phi, 0) }{\parteial \bgama}}
\newcommand{\lz}{  l_{z}(\bgama, \phi, \tau)}
\newcommand{\lo}{  l_{0}(\beta, \phi, 0)}
\newcommand{\E}{\mathbb{E}}
\newcommand{\dis}{\displaystyle}
\linespread{1.0}%espaço entre linhas
\begin{document}
%%%%%%%%%%%%%%%%%%%%%%%%%%%%%%%%%%%%%%%%%%%%%%%%%%%%%%%%
%                      CABEÇALHO                     %
%%%%%%%%%%%%%%%%%%%%%%%%%%%%%%%%%%%%%%%%%%%%%%%%%%%%%%%%
\begin{table}[H]
\centering
\begin{tabular*}{\textwidth}{l@{\extracolsep{\fill}}l@{\extracolsep{\fill}}}
\begin{tabular}[l]{@{}l@{}}\textbf{\orgao}\\\textbf{\instituto}\\\textbf{\departamento} \end{tabular} & \begin{tabular}[l]{@{}l@{}}\textbf{Professor: \professor}\\ {\email}\\ \textbf{Disciplina: \disciplina}\end{tabular}                                                       
\end{tabular*}
\end{table}
\begin{center}
\rule[2ex]{\textwidth}{1pt}\\
{\Large{\titulo}}
\end{center}
\rule[2ex]{\textwidth}{1pt}\\

\textbf{Assuntos:} Modelos Atômicos, Identificação dos Átomos, Tabela Periódica e Distribuição Eletrônica.

\begin{questions}[label=\protect\circled{\bfseries\arabic*}]
%%%%%%%%%%%%%%%%%%%%%%%%%%%%%%%%%%%%%%%%%%%%%%%%%%%%%%%%
%                      Questões                   %
%%%%%%%%%%%%%%%%%%%%%%%%%%%%%%%%%%%%%%%%%%%%%%%%%%%%%%%%

%=========================================================

\question  Thomson determinou, pela primeira vez, a relação entre a massa e a carga do elétron, o que pode ser
considerado como a descoberta do elétron. É reconhecida
como uma contribuição de Thomson ao modelo atômico,

\begin{partes}
	
\parte o átomo ser indivisível.
\parte a existência de partículas subatômicas.
\parte os elétrons ocuparem níveis discretos de energia.
\parte  os elétrons girarem em órbitas circulares ao redor do
núcleo.
\parte o átomo possuir um núcleo com carga positiva e uma
eletrosfera.

\end{partes}

\question Rutherford, ao fazer incidir partículas radioativas em lâmina metálica de ouro, observou que a maioria
das partículas atravessava a lâmina, algumas desviavam
e poucas refletiam. Identifique, dentre as afirmações a
seguir, aquela que não reflete as conclusões de Rutherford
sobre o átomo

\begin{partes} 

\parte Os átomos são esferas maciças e indestrutíveis.
\parte No átomo há grandes espaços vazios.
\parte No centro do átomo existe um núcleo pequeno e
denso.
\parte O núcleo do átomo tem carga positiva.
\parte Os elétrons giram ao redor do núcleo para equilibrar
a carga positiva.
\end{partes}

\question  Eletrosfera é a região do átomo que:

\begin{partes}

\parte concentra praticamente toda a massa do átomo.
\parte contém as partículas de carga elétrica positiva.
\parte possui partículas sem carga elétrica.
\parte permanece inalterada na formação dos íons.
\parte tem volume praticamente igual ao volume do átomo.

\end{partes}

\question Isótopos radioativos são empregados no diagnóstico e tratamento de inúmeras doenças. Qual é a principal propriedade que caracteriza um elemento químico?

\begin{partes}

\parte número de massa
\parte número de prótons
\parte número de nêutrons
\parte energia de ionização
\parte diferença entre o número de prótons e de nêutrons

\end{partes}

\question  Dentre as espécies químicas:

$$ B^{9}_{5}  \quad B^{10}_{5}  \quad B^{11}_{5} \qquad C^{10}_{6}  \quad C^{12}_{6} \quad C^{14}_{6} $$

as que representam átomos cujos núcleos possuem 6 nêutrons são:

\begin{partes}

\parte $C^{10}_{6} \quad  C^{12}_{6} $
\parte $B^{11}_{5} \quad  C^{12}_{6} $
\parte $B^{10}_{5}  \quad  B^{11}_{5}$
\parte $B^{9}_{5}  \quad  C^{14}_{6}$
\parte $C^{14}_{6}  \quad B^{10}_{5}$


\end{partes}

\question  Em um átomo com 22 elétrons e 26 nêutrons,
seu número atômico e número de massa são, respectivamente:

\begin{partes}

\parte 22 e 26 
\parte 48 e 22
 \parte 26 e 48  
 \parte 22 e 48
\parte  26 e 22


\end{partes}


\question O número de prótons, nêutrons e elétrons representados por
$\ce{^{138}_{56}Ba^{2+}} $ é, respectivamente:

\begin{partes}

\parte 56, 82 e 56 
\parte 82, 138 e 56
\parte  56, 82 e 54 
\parte 82, 194 e 56
 \parte 56, 82 e 58

\end{partes}


\question  Isótopos radioativos de iodo são utilizados no diagnóstico e tratamento de problemas da tireóide, e são,
em geral, ministrados na forma de sais de iodeto. O número de prótons, nêutrons e elétrons no isótopo 131 do $\ce{^{131}_{53}I^{-}} $ são, respectivamente:

\begin{partes}

\parte 53, 78 e 52
\parte 53, 78 e 54
 \parte  53, 131 e 53
 \parte 131, 53 e 131
 \parte  52, 78 e 53
\end{partes}


\question Um átomo possui 19 prótons, 20 nêutrons e 19 elétrons.
Qual dos seguintes átomos é seu isótono?

\begin{partes}

\parte $A^{21}_{19}$

\parte $B^{20}_{19}$

\parte $C^{38}_{18}$

\parte $D^{58}_{39}$

\parte $E^{39}_{20}$

\end{partes}

\question Analise as seguintes afirmativas:

I. Isótopos são átomos de um mesmo elemento que possuem mesmo número atômico e diferente número
de massa.

II. O número atômico de um elemento corresponde ao
número de prótons no núcleo de um átomo.

III. O número de massa corresponde à soma do número
de prótons e do número de elétrons de um elemento.

Está(ão) correta(s):

\begin{partes}
\parte apenas I.
\parte apenas II.
\parte apenas III.
\parte apenas I e II.
\parte apenas II e III

\end{partes}


\question Os implantes dentários estão mais seguros
no Brasil e já atendem às normas internacionais de qualidade. O grande salto de qualidade aconteceu no processo de confecção dos parafusos e pinos de titânio que
compõem as próteses. Feitas com ligas de titânio, essas
próteses são usadas para fixar coroas dentárias, aparelhos ortodônticos e dentaduras nos ossos da mandíbula
e do maxilar”. Jornal do Brasil, outubro, 1996.
Considerando que o número atômico do titânio é 22,
sua configuração eletrônica será:

\begin{partes}

\parte $1s^2\,2s^2\,2p^6\,3s^2\,3p^3$
\parte $1s^2\,2s^2\,2p^6\,3s^2\,3p^5$
\parte $1s^2\,2s^2\,2p^6\,3s^2\,3p^6\,4s^2$
\parte $1s^2\,2s^2\,2p^6\,3s^2\,3p^6\,4s^2\,3d^2$
\parte $1s^2\,2s^2\,2p^6\,3s^2\,3p^6\,4s^2\,3d^{10}\,4p^6$
\end{partes}


\question Um átomo cuja configuração ele-
trônica é $1s^2\,2s^2\,2p^6\,3s^2\,3p^6\,4s^2$ tem como número
atômico:

\begin{partes}

\parte 10 
\parte 18  
\parte 8
\parte  20 
\parte 2

\end{partes}

\question Qual é a distribuição eletrônica, em subníveis,
para o cátion $Ca^{2+}$? (Dado: Ca $=$ 20.)

\begin{partes}

\parte $1s^2\,2s^2\,2p^6\,3s^2\,3p^6\,4s^2$
\parte $1s^2\,2s^2\,3s^2\,3p^6\,3d^2$
\parte $1s^2\,2s^2\,2p^6\,3s^2\,3p^6$
\parte $1s^2\,2s^2\,2p^6\,3s^2\,3p^6\,4s^2\,3d^2$
\parte $1s^2\,2s^2\,3s^2\,3p^4\,4s^2$
\end{partes}

\question Um átomo, cujo número atômico é 18, está classificado na Tabela Periódica como:

\begin{partes}

\parte metal alcalino
\parte metal alcalino-terroso
\parte  metal terroso
\parte  ametal
\parte  gás nobre
\end{partes}

\question O número atômico do elemento que se encontra
no período III, família 3A é:

\begin{partes}

\item 10 
 \item 23 
 \item 31
 \item 12 
 \item 13

\end{partes}

\question Na classificação periódica, os elementos Ba
(grupo 2), Se (grupo 16) e Cl (grupo 17) são conhecidos, respectivamente, como:

\begin{partes}

\parte alcalino, halogênio e calcogênio
\parte alcalino-terroso, halogênio e calcogênio
\parte alcalino-terroso, calcogênio e halogênio
\parte alcalino, halogênio e gás nobre
\parte alcalino-terroso, calcogênio e gás nobre

\end{partes}

\question Ferro (Z = 26), manganês (Z = 25) e cromo
(Z = 24) são:

\begin{partes}
\parte metais alcalinos
\parte metais alcalinos-terrosos
\parte elementos de transição
\parte lantanídios
\parte calcogênios
\end{partes}


\question Um dos elementos químicos que tem se mostrado
muito eficiente no combate ao câncer de próstata é o
selênio (Se).
Com base na Tabela de Classificação Periódica dos Elementos, os símbolos de elementos com propriedades químicas semelhantes ao selênio são:

\begin{partes}
\parte Cl, Br, I 
\parte Te, S, Po 
\parte P, As, Sb 
\parte As, Br, Kr
\end{partes}


\question Um elemento neutro possui configuração eletrônica $1s^2\,2s^2\,2p^6\,3s^2\,3p^5$ Esse elemento é um:

\begin{partes}

\parte metal alcalino terroso.
\parte halogênio.
\parte metal do primeiro período de transição.
\parte gás nobre.
\parte elemento do grupo do nitrogênio.

\end{partes}


\question Uma distribuição eletrônica possível para
um elemento X, que pertence à mesma família do elemento bromo, cujo número atômico é igual a 35, é:

\begin{partes}

\parte $1s^2\,2s^2\,2p^5$
\parte $1s^2\,2s^2\,2p^6\,3s^2\,3p^1$
\parte  $1s^2\,2s^2\,2p^2$
\parte  $1s^2\,2s^2\,2p^6\,3s^1$
\parte $1s^2\,2s^2\,2p^6\,3s^2\,3p^6\,4s^2\,3d^5$

\end{partes}


\question \citep{ENEM(2017)} Um fato corriqueiro ao se cozinhar arroz é o derramamento de parte da água de
cozimento sobre a chama azul do fogo, mudando-a para uma chama amarela. Essa mudança de
cor pode suscitar interpretações diversas, relacionadas às substâncias presentes na água de
cozimento. Além do sal de cozinha (NaCl), nela se encontram carboidratos, proteínas e sais
minerais.

Cientificamente, sabe-se que essa mudança de cor da chama ocorre pela:

\begin{partes}

\parte reação do gás de cozinha com o sal, volatilizando gás cloro.
\parte  emissão de fótons pelo sódio, excitado por causa da chama.
\parte  produção de derivado amarelo, pela reação com o carboidrato.
\parte  reação do gás de cozinha com a água, formando gás hidrogênio.
\parte  excitação das moléculas de proteínas, com formação de luz amarela.


\end{partes}


\question \citep{ENEM(2018)}
a mitologia grega, Nióbia era a filha de Tântalo, dois personagens conhecidos pelo
sofrimento. O elemento químico de número atômico (Z) igual a 41 tem propriedades químicas e
físicas tão parecidas com as do elemento de número atômico 73 que chegaram a ser confundidos.
Por isso, em homenagem a esses dois personagens da mitologia grega, foi conferido a esses
elementos os nomes de nióbio (Z = 41) e tântalo (Z = 73). Esses dois elementos químicos adquiriram
grande importância econômica na metalurgia, na produção de supercondutores e em outras
aplicações na indústria de ponta, exatamente pelas propriedades químicas e físicas comuns aos dois.

A importância econômica e tecnológica desses elementos, pela similaridade de suas
propriedades químicas e físicas, deve-se a

\begin{partes}
\item terem elétrons no subnível f.
\item  serem elementos de transição interna.
\item  pertencerem ao mesmo grupo na tabela periódica.
\item  terem seus elétrons mais externos nos níveis 4 e 5, respectivamente.
\item  estarem localizados na família dos alcalinos terrosos e alcalinos, respectivamente.
\end{partes}





%================================================================
%-------------------FIM DA LISTA
%================================================================
\end{questions}
%===========================================================
%          BIBLIOGRAFIA
%===========================================================
\begin{thebibliography}{99}
\thispagestyle{empty}%myheadings
%===========================================================
%==============Livro
\bibitem[ENEM(2017)]{ENEM(2017)}
[1] \textbf{ENEM 2017}
(Exame Nacional do Ensino Médio).
\emph{INEP-Instituto Nacional de Estudos e Pesquisas Educacionais Anísio Teixeira}.{Ministério da Educação}. 

\bibitem[ENEM(2009)]{ENEM(2009)}
[2] \textbf{ENEM 2009}
(Exame Nacional do Ensino Médio).
\emph{INEP-Instituto Nacional de Estudos e Pesquisas Educacionais Anísio Teixeira}.{Ministério da Educação}. 

\bibitem[ENEM(2011)]{ENEM(2011)}
[3] \textbf{ENEM 2011}
(Exame Nacional do Ensino Médio).
\emph{INEP-Instituto Nacional de Estudos e Pesquisas Educacionais Anísio Teixeira}.{Ministério da Educação}. 

\bibitem[ENEM(2010)]{ENEM(2010)}
[4] \textbf{ENEM 2010}
(Exame Nacional do Ensino Médio).
\emph{INEP-Instituto Nacional de Estudos e Pesquisas Educacionais Anísio Teixeira}.{Ministério da Educação}. 

\bibitem[ENEM(2025)]{ENEM(2025)}
[5] \textbf{ENEM 2025}
(Exame Nacional do Ensino Médio).
\emph{INEP-Instituto Nacional de Estudos e Pesquisas Educacionais Anísio Teixeira}.{Ministério da Educação}.

\bibitem[ENEM(2024)]{ENEM(2024)}
[6] \textbf{ENEM 2024}
(Exame Nacional do Ensino Médio).
\emph{INEP-Instituto Nacional de Estudos e Pesquisas Educacionais Anísio Teixeira}.{Ministério da Educação}.

\bibitem[ENEM(2022)]{ENEM(2022)}
[7] \textbf{ENEM 2022}
(Exame Nacional do Ensino Médio).
\emph{INEP-Instituto Nacional de Estudos e Pesquisas Educacionais Anísio Teixeira}.{Ministério da Educação}.

\bibitem[ENEM(2020)]{ENEM(2020)}
[8] \textbf{ENEM 2020}
(Exame Nacional do Ensino Médio).
\emph{INEP-Instituto Nacional de Estudos e Pesquisas Educacionais Anísio Teixeira}.{Ministério da Educação}.

\bibitem[ENEM(2018)]{ENEM(2018)}
[9] \textbf{ENEM 2018}
(Exame Nacional do Ensino Médio).
\emph{INEP-Instituto Nacional de Estudos e Pesquisas Educacionais Anísio Teixeira}.{Ministério da Educação}.


%===========================================================
\end{thebibliography}
\end{document}
