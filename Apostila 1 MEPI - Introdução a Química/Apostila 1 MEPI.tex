% LISTA DE EXERCÍCIOS Template using "book"
% Created by Milena Lima 
%Email:milenascimentolima@gmail.com
% Science Project at school 2017-FAPEAM
% View: https://www.overleaf.com/read/syjxcffdygch
%=======================================================
%------------LISTA DE EXERCÍCIOS
%=======================================================
\documentclass[12pt,a4paper,oneside,openany]{book} 
%=======================================================
%-------------PACOTES 
%=======================================================
\usepackage[top=1cm,left=1cm,right=1.5cm,bottom=2cm]{geometry}
\usepackage[T1]{fontenc}%Especif. a codif. de caracteres
\usepackage{ae}%Auxílio para fontes e pdf
\usepackage[utf8]{inputenc}
\usepackage{lipsum}%Gerar Texto Aleatório
\usepackage[brazilian]{babel}%Traduzir para Português
\usepackage{indentfirst}%Faz indentações em parágrafos
\usepackage{graphicx}%Permite incluir figuras
\usepackage{subfig} %para criar sub figuras
\usepackage{float}% figuras
\usepackage{tabularx}
\usepackage{ragged2e}
\usepackage{multirow}
\usepackage[dvipsnames]{xcolor}%Admitir cores
\usepackage{amsmath,amssymb,amsthm}%Incluir expressões  matemáticas, equações, teoremas, símbolos, etc
\usepackage{lastpage}%Incluir Ficha catalográfica
\usepackage{epigraph}%Incluir Epígrafo
\usepackage{enumerate}
\usepackage{enumitem}
\newlist{questions}{enumerate}{3}
\setlist[questions]{label=\arabic*.}
\newcommand{\question}{\item}
\setlist[enumerate,1]{% (
leftmargin=*, itemsep=12pt, label={\textbf{\arabic*.)}}}
%---
\newlist{partes}{enumerate}{3}
\setlist[partes]{label=(\alph*)}
\newcommand{\parte}{\item}
%---
\newlist{subpartes}{enumerate}{3}
\setlist[subpartes]{label=\roman*)}
\newcommand{\subparte}{\item}
%---
\usepackage{array}
\usepackage{tikz}
\newcommand*\circled[1]{\tikz[baseline=(char.base)]{\node[shape=circle,draw,inner sep=2pt] (char) {#1};}}
\usepackage[sort&compress,round,comma,authoryear]{natbib}
\usepackage{makeidx}
\usepackage[colorlinks=true,urlcolor=magenta,citecolor=red,linkcolor=violet,bookmarks=true]{hyperref}
\usepackage{lscape}%Altera a orientação de uma página
\usepackage{pdflscape}
\usepackage{epstopdf} %converte figs eps em figs pdf
\usepackage{booktabs}
\usepackage{pdfpages}
\usepackage{textcomp}
\usepackage[many]{tcolorbox}
\usepackage{empheq}
\usepackage{tasks}%lista alfabética
\pagestyle{plain}
%================================================================
%------------DIGITE AQUI
%===============================================================
\newcommand{\orgao}{Universidade do Estado do Pará}
\newcommand{\instituto}{Movimento de Educação Popular e Inclusiva do Jurunas}
\newcommand{\departamento}{EEFM Arthur Porto}
\newcommand{\curso}{Ciências da Computação}
\newcommand{\professor}{Elias Sabát}
\newcommand{\disciplina}{Química}
\newcommand{\titulo}{1ª Lista de Exercícios: Introdução a Química}
\newcommand{\data}{\today}
\newcommand{\aluno}{ALUNO:}
\newcommand{\email}{{\bf Professor: Isaias Tobelem}}
\newcommand{\turma}{kkkkkkkkkkkkkkkkkkkkkkkkkk}
%===========================================================
\newcommand{\X}{\mathbf{X}}
\newcommand{\x}{\mathbf{x}}
\newcommand{\Z}{\mathbf{Z}}
\newcommand{\z}{\mathbf{z}}
\newcommand{\y}{\boldsymbol{y}}
\newcommand{\balpha}{\mbox{${ \bm \alpha}$}}
\newcommand{\bmu}{\mbox{${\bm \mu}$}}
\newcommand{\bbeta}{\mbox{${\bm \beta}$}}
\newcommand{\bteta}{\mbox{${\bm \theta}$}}
\newcommand{\bgama}{\mbox{${\bm \gamma}$}}
\newcommand{\bxi}{\mbox{${\bm \xl}$}}
\newcommand{\bvarphi}{\mbox{${ \bm \varphi}$}}
\newcommand{\SZ}{\mbox{ $Z$}}
\newcommand{\muz}{\mu_{z,l}}
\newcommand{\muo}{\mu_{0,l}}
\newcommand{\etao}{\eta_{0,l}}
\newcommand{\etaz}{\eta_{z,l}}
\newcommand{\xbeta}{x_{l}\bgama}
\newcommand{\mui}{\mu_{l}}
\newcommand{\zetaind}{\zeta \mathtt{I}_{\{s_{l} \in Z \}}} 
\newcommand{\spz}{ s_{l} \in z}
\newcommand{\snpz}{ s_{l} \notin z}
\newcommand{\sps}{ s_{l} \in S}
\newcommand{\gphi}{ \Gamma(\phi)}
\newcommand{\scan}{ \Lambda_{z}}
\newcommand{\gmuop}{ \Gamma(\muo\phi)}
\newcommand{\gmuzp}{ \Gamma(\muz \phi)}
\newcommand{\gumuop}{ \Gamma((1-\muo)\phi)}
\newcommand{\gumuzp}{ \Gamma((1-\muz)\phi)}
\newcommand{\dlobeta}{ \frac{\parteial l_{0}(\bgama, \phi, 0) }{\parteial \bgama}}
\newcommand{\lz}{  l_{z}(\bgama, \phi, \tau)}
\newcommand{\lo}{  l_{0}(\beta, \phi, 0)}
\newcommand{\E}{\mathbb{E}}
\newcommand{\dis}{\displaystyle}
\linespread{1.0}%espaço entre linhas
\begin{document}
%%%%%%%%%%%%%%%%%%%%%%%%%%%%%%%%%%%%%%%%%%%%%%%%%%%%%%%%
%                      CABEÇALHO                     %
%%%%%%%%%%%%%%%%%%%%%%%%%%%%%%%%%%%%%%%%%%%%%%%%%%%%%%%%
\begin{table}[H]
\centering
\begin{tabular*}{\textwidth}{l@{\extracolsep{\fill}}l@{\extracolsep{\fill}}}
\begin{tabular}[l]{@{}l@{}}\textbf{\orgao}\\\textbf{\instituto}\\\textbf{\departamento} \end{tabular} & \begin{tabular}[l]{@{}l@{}}\textbf{Professor: \professor}\\ {\email}\\ \textbf{Disciplina: \disciplina}\end{tabular}                                                       
\end{tabular*}
\end{table}
\begin{center}
\rule[2ex]{\textwidth}{1pt}\\
{\Large{\titulo}}
\end{center}
\rule[2ex]{\textwidth}{1pt}\\

\textbf{Assuntos:} Substância Pura, Misturas Homogêneas e Heterogêneas, Separação de Misturas, Substâncias Simples e Compostas e Mudanças de Estado Físico


\begin{questions}[label=\protect\circled{\bfseries\arabic*}]
%%%%%%%%%%%%%%%%%%%%%%%%%%%%%%%%%%%%%%%%%%%%%%%%%%%%%%%%
%                      Questões                   %
%%%%%%%%%%%%%%%%%%%%%%%%%%%%%%%%%%%%%%%%%%%%%%%%%%%%%%%%

%=========================================================

\question Em um laboratório de Química foram preparadas as seguintes misturas:

I. água /gasolina

II. água/sal

III. água/areia

IV. gasolina/sal

V. gasolina/areia

Quais dessas misturas são homogêneas?
\begin{partes}
    \parte  Nenhuma. 
   \parte II e III. 
   \parte II e IV.
\parte Somente II. 
\parte  I e II.
\end{partes}

\question Misturando, agitando bem e deixando um certo tempo
em repouso, diga quantas fases surgirão em cada um
dos sistemas:

\begin{partes}
    \parte água e álcool
\parte  água e éter
\parte água, álcool e acetona
\parte  água, álcool e mercúrio
\parte  água, gasolina e areia
\end{partes}


\question Considere as misturas:

I. areia e água

II. sangue

III. água e acetona

IV. iodo dissolvido em álcool etílico

Classificam-se como homogêneas:
\begin{partes}
\parte apenas I e II.
\parte apenas I e III.
\parte apenas II e IV.
\parte apenas III e IV. 
\parte  apenas I, II e III.    
\end{partes}

\question O naftaleno, comercialmente conhecido como
naftalina, empregado para evitar baratas em roupas, funde
em temperaturas superiores a 80 °C. Sabe-se que bolinhas de naftalina, à temperatura ambiente, têm suas
massas constantemente diminuídas, terminando por desaparecer sem deixar resíduo. Essa observação pode ser
explicada pelo fenômeno da:

\begin{partes}
    \parte fusão.
\parte  sublimação.
\parte  solidificação.
\parte  liquefação.
\parte ebulição.
\end{partes}

\question Resfriando-se progressivamente água destilada, quando começar a passagem do estado líquido para
o sólido, a temperatura:

\begin{partes}
   \parte permanecerá constante, enquanto houver líquido presente.
\parte permanecerá constante, sendo igual ao ponto de
condensação da substância.
\parte diminuirá gradativamente.
\parte permanecerá constante, mesmo depois de todo líquido desaparecer.
\parte aumentará gradativamente.
\end{partes}

\question Em um sistema, bem misturado, constituído de
areia, sal, açúcar, água e gasolina, o número de fases é:

\begin{partes}
    \parte 2 
    \parte 3 
    \parte 4 
    \parte 5 
    \parte 6
\end{partes}

\question O aquecimento global já apresenta sinais visíveis em alguns pontos do planeta. Numa ilha do Alasca,
na Aldeia de Shishmaret, por exemplo, as geleiras já demoram mais a congelar, no inverno; descongelam mais
rápido, na primavera, e há mais icebergs. Desde 1971, a temperatura aumentou, em média, 2 °C.
As mudanças de estados descritas no texto, são, respectivamente:

\begin{partes}
    \parte solidificação e fusão.
\parte solidificação e condensação.
\parte sublimação e solidificação.
\parte solidificação e ebulição.
\parte fusão e condensação.
\end{partes}


\parte  Necessitou-se retirar o conteúdo do tanque de combustível de um carro. Para isso, fez-se sucção
com um pedaço de mangueira introduzido no tanque,
deixando-se escorrer o líquido para um recipiente colocado no chão. Esse processo é chamado de:

\begin{partes}
    \parte  decantação 
   \parte sifonação 
   \parte destilação
\parte filtração 
\parte centrifugação
\end{partes}

\question Considere uma substância cuja fórmula é $H_{3}PO_{4}$.
Essa substância é composta por:

\begin{partes}
    \parte  2 elementos 
    \parte 8 elementos
\parte 3 elementos
\parte 4 elementos
\end{partes}

\question Qual das alternativas abaixo contém somente
substâncias simples:

\begin{partes}
    \parte $H_{2}O$, HCl, CaO 
    \parte Au, Fe, $O_{2}$
\parte $H_{2}O$, Au, K 
 \parte $H_{2}$, $Cl_{2}$, NaK
\parte $H_{2}O$, $Cl_{2}$, K
\end{partes}

\question Água mineral engarrafada, propanona
($C_{3}H_{6}O$) e gás oxigênio são classificados, respectivamente, como:

\begin{partes}
    \parte substância pura composta, substância pura simples e
mistura homogênea.
\parte substância pura composta, mistura homogênea e substância pura simples.
\parte mistura heterogênea, substância pura simples e substância pura simples.
\parte mistura homogênea,substância pura composta e substância pura composta.
\parte mistura homogênea,substância pura composta e substância pura simples.
\end{partes}


\question
\citep{ ENEM(2009)} O ciclo da água é fundamental para a preservação da vida no planeta. As condições climáticas
da Terra permitem que a água sofra mudanças
de fase e a compreensão dessas transformações é fundamental para se entender o ciclo
hidrológico. Numa dessas mudanças, a água ou
a umidade da terra absorve o calor do sol e dos
arredores. Quando já foi absorvido calor suficiente, algumas das moléculas do líquido podem
ter energia necessária para começar a subir
para a atmosfera. A transformação mencionada no texto é a:
\begin{partes}
\parte fusão.
\parte liquefação.
\parte evaporação.
\parte solidificação.
\parte condensação.
\end{partes}

\question
\citep{ ENEM(2009)} A ciência propõe formas de explicar a natureza
e seus fenômenos que, muitas vezes, confrontam o conhecimento popular ou o senso comum. Um bom exemplo desse descompasso é
a explicação microscópica da flutuação do gelo
na água. Do ponto de vista atômico, podem-se
representar os três estados físicos dessa substância como nas figuras a seguir, nas quais as
bolas representam as moléculas de água.

\begin{figure}[H]
    \centering
    \includegraphics[width=0.5\linewidth]{1.png}
    
\end{figure}

Considerando-se as representações das moléculas de água nos três estados físicos e seu
comportamento anômalo, é correto afirmar que:

\begin{partes}
    \parte sólidos afundam na água.
\parte a interação entre as moléculas está restrita
ao estado sólido.
\parte a figura B é a que melhor representa a água
no estado líquido.
\parte  a figura A é a que melhor representa o gelo,
ou seja, água no estado sólido.
\parte aumenta a distância entre as moléculas da
substância à medida que a temperatura aumenta.
\end{partes}


\question
\citep{ ENEM(2013)} Entre as substâncias usadas para o
tratamento de água está o sulfato de alumínio que, em meio
alcalino, forma partículas em suspensão na água, às quais as
impurezas presentes no meio aderem.
O método de separação comumente usado para retirar o
sulfato de alumínio com as impurezas aderidas é a:

\begin{partes}
    \parte flotação.
\parte levigação.
\parte ventilação.
\parte peneiração.
\parte centrifugação.
\end{partes}

\question 
\citep{ ENEM(2011)} Belém é cercada por 39 ilhas, e suas
populações convivem com ameaças de doenças. O motivo,
apontado por especialistas, é a poluição da água do rio,
principal fonte de sobrevivência dos ribeirinhos. A diarreia
é frequente nas crianças e ocorre como consequência da
falta de saneamento básico, já que a população não tem
acesso à água de boa qualidade. Como não há água potável,
a alternativa é consumir a do rio.
O procedimento adequado para tratar a água dos rios, a fim
de atenuar os problemas de saúde causados por
microrganismos a essas populações ribeirinhas é a:

\begin{partes}
    \parte filtração.
\parte cloração.
\parte coagulação.
\parte fluoretação.
\parte decantação. 
\end{partes}

\question
\citep{ENEM(2010)}
Em visita a uma usina sucroalcooleira,
um grupo de alunos pôde observar a série de processos de
beneficiamento da cana-de-açúcar, entre os quais se
destacam:

1. A cana chega cortada da lavoura por meio de caminhões
e é despejada em mesas alimentadoras que a conduzem
para as moendas. Antes de ser esmagada para a retirada
do caldo açucarado, toda a cana é transportada por
esteiras e passada por um eletroímã para a retirada de
materiais metálicos.

2. Após se esmagar a cana, o bagaço segue para as
caldeiras, que geram vapor e energia para toda a usina.

3. O caldo primário, resultante do esmagamento, é passado
por filtros e sofre tratamento para transformar-se em
açúcar refinado e etanol.

Com base nos destaques da observação dos alunos, quais
operações físicas de separação de materiais foram
realizadas nas etapas de beneficiamento da cana-de açúcar? 

\begin{partes}
    \parte Separação mecânica, extração, decantação.
\parte Separação magnética, combustão, filtração.
\parte Separação magnética, extração, filtração.
\parte Imantação, combustão, peneiração.
\parte  Imantação, destilação, filtração. 
\end{partes}

\question
\citep{ENEM(2025)} No início do século XX, as fórmulas das substâncias eram representadas de modo diferente do atual. A figura apresenta uma
fotografia bem antiga (1909) que registra uma aula de química ministrada em um colégio em Santos (SP). Um olhar mais atento
permite identificar como os compostos químicos eram representados.

\begin{figure}[H]
    \centering
    \includegraphics[width=0.5\linewidth]{2.png}
\end{figure}

O nitrogênio era chamado de azoto e representado pelo símbolo Az. Vê-se na lousa a equação representativa da adição do
oxigênio atômico (O) ao monóxido de nitrogênio (AzO) com a formação de dióxido de nitrogênio ($AzO^2$). Analogamente, o nitrato
de sódio era representado por $NaAzO^3$. 

Em 1909, as representações das substâncias ácido nítrico e cloreto de cálcio, tendo por base essas informações e seguindo a
mesma lógica, seriam, respectivamente:

\begin{partes}
    \parte $HAzO^3$ e $CaCl^2$
\parte $HAz^3O$ e $Ca^2Cl$
\parte $H^3AzO^4$ e CaCl
\parte $HAz^3O$ e $KCl^2$
\parte $HAzO2$ e KCl
\end{partes}

 \question  \citep{ENEM(2024)} Existe um processo de purificação de água em que são
removidos os sais dissolvidos. A água que passa por esse processo
é muito utilizada em laboratórios de química, em indústrias
(como solvente), em baterias de carros etc. Entretanto, esse
tipo de água não é adequado para ingestão, pois pode causar
problemas de saúde, como carência iônica e diarreia.

Essa água é chamada de:

\begin{partes}
    \parte dura.
\parte pesada.
\parte sanitária.
\parte destilada.
\parte oxigenada.
\end{partes}

\question  \citep{ENEM(2024)} O magnésio metálico utilizado em ligas leves é
produzido em um processo que envolve várias etapas e
utiliza água do mar como matéria-prima. A primeira etapa
desse processo consiste na reação entre o íon $Mg^{2+}$ e
hidróxido de cálcio, $Ca(OH)_{2}$, obtendo uma mistura que
contém hidróxido de magnésio, pouco solúvel, e íons $Ca^{2+}$,
de acordo com a equação química:

$$Mg^{2+}_{(aq)} + Ca(OH)_{2(aq)} \longrightarrow  Mg(OH)_{2(s)} + Ca^{2+}_{(aq)}$$

O método adequado para separar o $Mg(OH)_2$ dessa
mistura é a: 

\begin{partes}
    \parte filtração.
\parte catação.
\parte destilação.
\parte dissolução.
\parte evaporação.
\end{partes}

\question \citep{ENEM(2022)}
A água bruta coletada de mananciais apresenta
alto índice de sólidos suspensos, o que a deixa com
um aspecto turvo. Para se obter uma água límpida e
potável, ela deve passar por um processo de purificação
numa estação de tratamento de água. Nesse processo,
as principais etapas são, nesta ordem: coagulação,
decantação, filtração, desinfecção e fluoretação.

Qual é a etapa de retirada de grande parte desses
sólidos?

\begin{partes}
    \parte Coagulação.
\parte Decantação.
\parte Filtração.
\parte Desinfecção.
\parte Fluoretação.
\end{partes}

\question \citep{ENEM(2022)}
 O urânio é empregado como fonte de energia em
reatores nucleares. Para tanto, o seu mineral deve
ser refinado, convertido a hexafluoreto de urânio e
posteriormente enriquecido, para aumentar de $0,7 \%$ a $3 \%$
a abundância de um isótopo específico — o urânio-235.
Uma das formas de enriquecimento utiliza a pequena
diferença de massa entre os hexafluoretos de urânio-235 e
de urânio-238 para separá-los por efusão, precedida pela
vaporização. Esses vapores devem efundir repetidamente
milhares de vezes através de barreiras porosas formadas por
telas com grande número de pequenos orifícios. No entanto,
devido à complexidade e à grande quantidade de energia
envolvida, cientistas e engenheiros continuam a pesquisar
procedimentos alternativos de enriquecimento.

Considerando a diferença de massa mencionada entre os
dois isótopos, que tipo de procedimento alternativo ao da
efusão pode ser empregado para tal finalidade?

\begin{partes}
    \parte Peneiração.
\parte Centrifugação.
\parte Extração por solvente.
\parte Destilação fracionada.
\parte Separação magnética.
\end{partes}

\question \citep{ENEM(2020)} A obtenção de óleos vegetais, de maneira geral, passa pelas etapas descritas no quadro. 

\begin{figure}[H]
    \centering
    \includegraphics[width=0.9\linewidth]{3.png}
\end{figure}

Qual das subetapas do processo é realizada em função apenas da polaridade das substâncias?

\begin{partes}
    \parte Trituração.
\parte Cozimento.
\parte Prensagem.
\parte Extração.
\parte Destilação.
\end{partes}

%================================================================
%-------------------FIM DA LISTA
%================================================================
\end{questions}
%===========================================================
%          BIBLIOGRAFIA
%===========================================================
\begin{thebibliography}{99}
\thispagestyle{empty}%myheadings
%===========================================================
%==============Livro
\bibitem[ENEM(2013)]{ENEM(2013)}
[1] \textbf{ENEM 2013}
(Exame Nacional do Ensino Médio).
\emph{INEP-Instituto Nacional de Estudos e Pesquisas Educacionais Anísio Teixeira}.{Ministério da Educação}. 

\bibitem[ENEM(2009)]{ENEM(2009)}
[2] \textbf{ENEM 2009}
(Exame Nacional do Ensino Médio).
\emph{INEP-Instituto Nacional de Estudos e Pesquisas Educacionais Anísio Teixeira}.{Ministério da Educação}. 

\bibitem[ENEM(2011)]{ENEM(2011)}
[3] \textbf{ENEM 2011}
(Exame Nacional do Ensino Médio).
\emph{INEP-Instituto Nacional de Estudos e Pesquisas Educacionais Anísio Teixeira}.{Ministério da Educação}. 

\bibitem[ENEM(2010)]{ENEM(2010)}
[4] \textbf{ENEM 2010}
(Exame Nacional do Ensino Médio).
\emph{INEP-Instituto Nacional de Estudos e Pesquisas Educacionais Anísio Teixeira}.{Ministério da Educação}. 

\bibitem[ENEM(2025)]{ENEM(2025)}
[5] \textbf{ENEM 2025}
(Exame Nacional do Ensino Médio).
\emph{INEP-Instituto Nacional de Estudos e Pesquisas Educacionais Anísio Teixeira}.{Ministério da Educação}.

\bibitem[ENEM(2024)]{ENEM(2024)}
[6] \textbf{ENEM 2024}
(Exame Nacional do Ensino Médio).
\emph{INEP-Instituto Nacional de Estudos e Pesquisas Educacionais Anísio Teixeira}.{Ministério da Educação}.

\bibitem[ENEM(2022)]{ENEM(2022)}
[7] \textbf{ENEM 2022}
(Exame Nacional do Ensino Médio).
\emph{INEP-Instituto Nacional de Estudos e Pesquisas Educacionais Anísio Teixeira}.{Ministério da Educação}.

\bibitem[ENEM(2020)]{ENEM(2020)}
[7] \textbf{ENEM 2020}
(Exame Nacional do Ensino Médio).
\emph{INEP-Instituto Nacional de Estudos e Pesquisas Educacionais Anísio Teixeira}.{Ministério da Educação}.


%===========================================================
\end{thebibliography}
\end{document}
