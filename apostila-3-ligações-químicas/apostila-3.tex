% LISTA DE EXERCÍCIOS Template using "book"
% Created by Milena Lima 
%Email:milenascimentolima@gmail.com
% Science Project at school 2017-FAPEAM
% View: https://www.overleaf.com/read/syjxcffdygch
%=======================================================
%------------LISTA DE EXERCÍCIOS
%=======================================================
\documentclass[12pt,a4paper,oneside,openany]{book} 
%=======================================================
%-------------PACOTES 
%=======================================================
\usepackage[top=1cm,left=1cm,right=1.5cm,bottom=2cm]{geometry}
\usepackage[T1]{fontenc}%Especif. a codif. de caracteres
\usepackage{ae}%Auxílio para fontes e pdf
\usepackage[utf8]{inputenc}
\usepackage{lipsum}%Gerar Texto Aleatório
\usepackage[brazil]{babel}%Traduzir para Português
\usepackage{indentfirst}%Faz indentações em parágrafos
\usepackage{graphicx}%Permite incluir figuras
\usepackage{subfig} %para criar sub figuras
\usepackage{float}% figuras
\usepackage{tabularx}
\usepackage{ragged2e}
\usepackage{multirow}
\usepackage[dvipsnames]{xcolor}%Admitir cores
\usepackage{amsmath,amssymb,amsthm}%Incluir expressões  matemáticas, equações, teoremas, símbolos, etc
\usepackage{lastpage}%Incluir Ficha catalográfica
\usepackage{epigraph}%Incluir Epígrafo
\usepackage{enumerate}
\usepackage{enumitem}
\newlist{questions}{enumerate}{3}
\setlist[questions]{label=\arabic*.}
\newcommand{\question}{\item}
\setlist[enumerate,1]{% (
leftmargin=*, itemsep=12pt, label={\textbf{\arabic*.)}}}
%---
\newlist{partes}{enumerate}{3}
\setlist[partes]{label=(\alph*)}
\newcommand{\parte}{\item}
%---
\newlist{subpartes}{enumerate}{3}
\setlist[subpartes]{label=\roman*)}
\newcommand{\subparte}{\item}
%---
\usepackage{array}
\usepackage{tikz}
\newcommand*\circled[1]{\tikz[baseline=(char.base)]{\node[shape=circle,draw,inner sep=2pt] (char) {#1};}}
\usepackage[sort&compress,round,comma,authoryear]{natbib}
\usepackage{makeidx}
\usepackage[colorlinks=true,urlcolor=magenta,citecolor=red,linkcolor=violet,bookmarks=true]{hyperref}
\usepackage{lscape}%Altera a orientação de uma página
\usepackage{pdflscape}
\usepackage{epstopdf} %converte figs eps em figs pdf
\usepackage{booktabs}
\usepackage{pdfpages}
\usepackage{textcomp}
\usepackage[many]{tcolorbox}
\usepackage{empheq}
\usepackage{tasks}%lista alfabética
\pagestyle{plain}
\usepackage{mhchem}
%================================================================
%------------DIGITE AQUI
%===============================================================
\newcommand{\orgao}{Universidade do Estado do Pará}
\newcommand{\instituto}{Movimento de Educação Popular e Inclusiva do Jurunas}
\newcommand{\departamento}{EEFM Arthur Porto}
\newcommand{\curso}{Ciências da Computação}
\newcommand{\professor}{Elias Sabát}
\newcommand{\disciplina}{Química}
\newcommand{\titulo}{3\textsuperscript{a} Lista de Exercícios: Ligações Químicas}
\newcommand{\data}{\today}
\newcommand{\aluno}{ALUNO:}
\newcommand{\email}{{\bf Professor: Isaias Tobelem}}
\newcommand{\turma}{kkkkkkkkkkkkkkkkkkkkkkkkkk}
%===========================================================
\newcommand{\X}{\mathbf{X}}
\newcommand{\x}{\mathbf{x}}
\newcommand{\Z}{\mathbf{Z}}
\newcommand{\z}{\mathbf{z}}
\newcommand{\y}{\boldsymbol{y}}
\newcommand{\balpha}{\mbox{${ \bm \alpha}$}}
\newcommand{\bmu}{\mbox{${\bm \mu}$}}
\newcommand{\bbeta}{\mbox{${\bm \beta}$}}
\newcommand{\bteta}{\mbox{${\bm \theta}$}}
\newcommand{\bgama}{\mbox{${\bm \gamma}$}}
\newcommand{\bxi}{\mbox{${\bm \xl}$}}
\newcommand{\bvarphi}{\mbox{${ \bm \varphi}$}}
\newcommand{\SZ}{\mbox{ $Z$}}
\newcommand{\muz}{\mu_{z,l}}
\newcommand{\muo}{\mu_{0,l}}
\newcommand{\etao}{\eta_{0,l}}
\newcommand{\etaz}{\eta_{z,l}}
\newcommand{\xbeta}{x_{l}\bgama}
\newcommand{\mui}{\mu_{l}}
\newcommand{\zetaind}{\zeta \mathtt{I}_{\{s_{l} \in Z \}}} 
\newcommand{\spz}{ s_{l} \in z}
\newcommand{\snpz}{ s_{l} \notin z}
\newcommand{\sps}{ s_{l} \in S}
\newcommand{\gphi}{ \Gamma(\phi)}
\newcommand{\scan}{ \Lambda_{z}}
\newcommand{\gmuop}{ \Gamma(\muo\phi)}
\newcommand{\gmuzp}{ \Gamma(\muz \phi)}
\newcommand{\gumuop}{ \Gamma((1-\muo)\phi)}
\newcommand{\gumuzp}{ \Gamma((1-\muz)\phi)}
\newcommand{\dlobeta}{ \frac{\parteial l_{0}(\bgama, \phi, 0) }{\parteial \bgama}}
\newcommand{\lz}{  l_{z}(\bgama, \phi, \tau)}
\newcommand{\lo}{  l_{0}(\beta, \phi, 0)}
\newcommand{\E}{\mathbb{E}}
\newcommand{\dis}{\displaystyle}
\linespread{1.0}%espaço entre linhas
\begin{document}
%%%%%%%%%%%%%%%%%%%%%%%%%%%%%%%%%%%%%%%%%%%%%%%%%%%%%%%%
%                      CABEÇALHO                     %
%%%%%%%%%%%%%%%%%%%%%%%%%%%%%%%%%%%%%%%%%%%%%%%%%%%%%%%%
\begin{table}[H]
\centering
\begin{tabular*}{\textwidth}{l@{\extracolsep{\fill}}l@{\extracolsep{\fill}}}
\begin{tabular}[l]{@{}l@{}}\textbf{\orgao}\\\textbf{\instituto}\\\textbf{\departamento} \end{tabular} & \begin{tabular}[l]{@{}l@{}}\textbf{Professor: \professor}\\ {\email}\\ \textbf{Disciplina: \disciplina}\end{tabular}                                                       
\end{tabular*}
\end{table}
\begin{center}
\rule[2ex]{\textwidth}{1pt}\\
{\Large{\titulo}}
\end{center}
\rule[2ex]{\textwidth}{1pt}\\

\textbf{Assuntos:} Ligações Iônicas, Covalentes e Metálicas, Estrutura de Lewis, Geometria Molecular, Polaridade e Forças Intermoleculares

\begin{questions}[label=\protect\circled{\bfseries\arabic*}]
%%%%%%%%%%%%%%%%%%%%%%%%%%%%%%%%%%%%%%%%%%%%%%%%%%%%%%%%
%                      Questões                   %
%%%%%%%%%%%%%%%%%%%%%%%%%%%%%%%%%%%%%%%%%%%%%%%%%%%%%%%%

%=========================================================

\question Para que um átomo neutro de cálcio se transfor-
me no íon $Ca^{2+}$, ele deve:

\begin{partes}

 \parte receber dois elétrons
 \parte perder dois prótons
 \parte receber dois prótons
 \parte perder um próton
 \parte perder dois elétrons

\end{partes}


\question Em um composto, sendo A o cátion, B o ânion e $A_{3}B_{2}$ a
fórmula, provavelmente os átomos A e B, no estado normal, tinham, respectivamente, os seguintes números de
elétrons periféricos: 

\begin{partes}

\parte 3 e 2 
\parte 3 e 6
\parte  2 e 3 
\parte  5 e 6
\parte  2 e 5

\end{partes}

\question Dados: O (Z = 8); C (Z = 6); F (Z = 9);
H (Z = 1).

A molécula que apresenta somente uma ligação covalente
normal é:

\begin{partes}

  \parte $F_{2}$

  \parte  $O_{3}$

  \parte  $O_{2}$

  \parte  $H_{2}O$

  \parte $CO$

\end{partes}

\question  Ao formar ligações covalentes
com o hidrogênio, a eletrosfera do silício adquire configuração de gás nobre. Com isso, é de se esperar a formação da molécula: 

\begin{partes}

\parte $SiH$ 
\parte  $SiH_{2}$
\parte  $SiH_{3}$  
\parte  $SiH_{4}$
\parte $SiH_{5}$

\end{partes}


\question Os elementos X e Y têm, respectivamente, 2 e
6 elétrons na camada de valência. Quando X e Y reagem,
forma-se um composto:

\begin{partes}

\parte covalente, de fórmula XY
\parte covalente, de fórmula $XY_{2}$
\parte covalente, de fórmula $X_{2}Y_{3}$
\parte iônico, de fórmula XY 

\end{partes}


\question Quando se comparam as espécies químicas
$CH_{4}$, $NH_{3}$ e NaCl, pode-se afirmar que os átomos estão
unidos por ligações covalentes somente no:

\begin{partes}

\parte $CH_{4}$ e no $NH_{3}$
\parte $NH_{3}$ e no NaCl
\parte $CH_{4}$ e no NaCl
\parte $CH_{4}$
\parte $NH_{3}$

\end{partes}


\question Dentre os compostos a seguir, indique qual deles apresenta apenas ligações covalentes normais:

\begin{partes}

\parte $SO_{3}$
\parte NaCl
\parte $NH_{3}$
\parte $O_{3}$
\parte $H_{2}SO_{4}$

\end{partes}


\question As propriedades ductilidade, maleabilidade,
brilho e condutividade elétrica caracterizam:

\begin{partes}
 
\parte cloreto de potássio e alumínio
\parte cobre e prata
\parte talco e mercúrio
\parte grafita e diamante
\parte aço e P.V.C.

\end{partes}


\question  Entre as substâncias simples puras constituídas por
átomos de S, As, Cd, I e Br, a que deve conduzir melhor a
corrente elétrica é a substância:

\begin{partes}


\parte enxofre 
\parte cádmio 
\parte bromo
\parte arsênio 
\parte iodo
\end{partes}


\question  Qual das substâncias a seguir tem molécula linear e apresenta ligações duplas?

\begin{partes}

\parte HCl
\parte $H_{2}O$
\parte $N_{2}$
\parte $CO_{2}$
\parte $NH_{3}$ 

\end{partes}

\question Assinale a alternativa em que não há exata correspondência entre a molécula e sua forma
geométrica:

\begin{partes}

\parte $N_{2}$ - Linear
\parte $CO_{2}$ - Linear
\parte $H_{2}O$- Angular
\parte $CCl_{4}$ - Tetraédrica
\parte $BF_{3}$ - Pirâmide trigonal

\end{partes}


\question Com relação à geometria das moléculas, a opção correta a seguir é:

\begin{partes}

 \parte NO - linear, $CO_{2}$ - linear, $NF_{3}$ - piramidal, $H_{2}O$ - angular, $BF_{3}$ - trigonal plana.
\parte NO - linear, $CO_{2}$ - angular, $NF_{3}$  - piramidal, $H_{2}O$ - angular, $BF_{3}$ - trigonal plana.
\parte NO - linear, $CO_{2}$  - trigonal, $NF_{3}$ - trigonal, $H_{2}O$ - linear, $BF_{3}$ - piramidal.
\parte NO - angular, $CO_{2}$  - linear, $NF_{3}$  - piramidal, $H_{2}O$ - angular, $BF_{3}$ - trigonal.
\parte NO - angular, $CO_{2}$ - trigonal, $NF_{3}$  - trigonal, $H_{2}O$ - linear, $BF_{3}$ - piramidal.

\end{partes}


\question Analise as seguintes informações:

I. A molécula $CO_{2}$ é apolar, sendo formada por ligações covalentes polares.

II. A molécula $H_{2}O$ é polar, sendo formada por ligações covalentes apolares.

III. A molécula $NH_{3}$ é polar, sendo formada por ligações iônicas.

Concluiu-se que:

\begin{partes}

\parte  somente I é correta.
\parte somente II é correta.
\parte somente III é correta.
\parte somente II e III são corretas.
\parte somente I e III são corretas.

\end{partes}


\question entre as afirmativas abaixo, assinalar a que contém a afirmação incorreta.

\begin{partes}

\parte Ligação covalente é aquela que se dá pelo compartilhamento de elétrons entre dois átomos.

\parte O composto covalente HCl é polar, devido à diferença de eletronegatividade existente entre os átomos de
hidrogênio e cloro.

\parte O composto formado entre um metal alcalino e halogênio é covalente.

\parte A substância da fórmula $Br_{2}$ é apolar.

\parte  A substância da fórmula $CaI_{2}$ é iônica.

\end{partes}


\question As substâncias $SO_{2}$ e $CO_{2}$ apresentam moléculas que possuem ligações polarizadas. Sobre as
moléculas destas substâncias é correto afirmar se que:

\begin{partes}

\parte  ambas são polares, pois apresentam ligações polarizadas.
\parte ambas são apolares, pois apresentam geometria linear.
\parte apenas o $CO_{2}$ é apolar, pois apresenta geometria linear.
\parte ambas são polares, pois apresentam geometria angular.
\parte apenas o $SO_{2}$ é apolar, pois apresenta geometria linear

\end{partes}


\question As ligações químicas nas substâncias K, HCl, KCl e $Cl_{2}$, são respectivamente:

\begin{partes}

\parte metálica, covalente polar, iônica, covalente apolar.
\parte iônica, covalente polar, metálica, covalente apolar.
\parte covalente apolar, covalente polar, metálica, covalente apolar.
\parte metálica, covalente apolar, iônica, covalente polar.
\parte covalente apolar, covalente polar, iônica, metálica.
\end{partes}


\question Os compostos FeO, NO, $F_{2}$ , NaCl e HCl apresentam, respectivamente, os seguintes tipos de
ligações:

 \begin{partes}

\parte iônica, covalente apolar, metálica, iônica e covalente polar.
\parte covalente polar, covalente polar, covalente apolar, iônica e molecular.
\parte metálica, iônica, covalente pura, molecular e iônica.
\parte iônica, covalente polar, covalente apolar, iônica e covalente polar.
\parte iônica, covalente apolar, covalente apolar, iônica e iônica.

\end{partes}





\question \citep{ENEM(2012)}
fosfatidilserina é um fosfolipídio aniônico cuja interação com cálcio livre regula processos
de transdução celular e vem sendo estudada no desenvolvimento de biossensores nanométricos. A
figura representa a estrutura da fosfatidilserina:
\begin{figure}[H]
    \centering
    \includegraphics[width=0.8\linewidth]{1.png}
\end{figure}
Com base nas informações do texto, a natureza da interação da fosfatidilserina com o cálcio
livre é do tipo:

\begin{partes}

\parte iônica somente com o grupo aniônico fosfato, já que o cálcio livre é um cátion
monovalente.
\parte iônica com o cátion amônio, porque o cálcio livre é representado como um ânion
monovalente.
\parte iônica com os grupos aniônicos fosfato e carboxila, porque o cálcio em sua forma livre
é um cátion divalente.
\parte covalente com qualquer dos grupos não carregados da fosfatidilserina, uma vez que
estes podem doar elétrons ao cálcio livre para formar a ligação.
\parte covalente com qualquer grupo catiônico da fosfatidilserina, visto que o cálcio na sua
forma livre poderá compartilhar seus elétrons com tais grupos.

\end{partes}


\question \citep{ENEM(2014)}
As propriedades físicas e químicas de uma certa substância estão relacionadas às interações
entre as unidades que a constituem, isto é, as ligações químicas entre átomos ou íons e as forças
intermoleculares que a compõem. No quadro, estão relacionadas algumas propriedades de cinco
substâncias.
\begin{figure}[H]
    \centering
    \includegraphics[width=0.8\linewidth]{2.png}
\end{figure}
Qual substância apresenta propriedades que caracterizam o cloreto de sódio (NaCl)?

\begin{partes}
\parte I
 \parte II
\parte III
\parte IV
\parte V
\end{partes}

\question \citep{ENEM(2013)}
A palha de aço, um material de baixo custo e vida útil pequena, utilizada para lavar louças, é
um emaranhado de fios leves e finos que servem para a remoção por atrito dos resíduos aderidos
aos objetos.

A propriedade do aço que justifica o aspecto físico descrito no texto é a

\begin{partes}

\parte ductilidade.

 \parte maleabilidade.

\parte densidade baixa.

\parte condutividade elétrica.

\parte condutividade térmica.

\end{partes}



%================================================================
%-------------------FIM DA LISTA
%================================================================
\end{questions}
%===========================================================
%          BIBLIOGRAFIA
%===========================================================
\begin{thebibliography}{99}
\thispagestyle{empty}%myheadings
%===========================================================
%==============Livro
\bibitem[ENEM(2013)]{ENEM(2013)}
[1] \textbf{ENEM 2013}
(Exame Nacional do Ensino Médio).
\emph{INEP-Instituto Nacional de Estudos e Pesquisas Educacionais Anísio Teixeira}.{Ministério da Educação}. 

\bibitem[ENEM(2009)]{ENEM(2009)}
[2] \textbf{ENEM 2009}
(Exame Nacional do Ensino Médio).
\emph{INEP-Instituto Nacional de Estudos e Pesquisas Educacionais Anísio Teixeira}.{Ministério da Educação}. 

\bibitem[ENEM(2011)]{ENEM(2011)}
[3] \textbf{ENEM 2011}
(Exame Nacional do Ensino Médio).
\emph{INEP-Instituto Nacional de Estudos e Pesquisas Educacionais Anísio Teixeira}.{Ministério da Educação}. 

\bibitem[ENEM(2010)]{ENEM(2010)}
[4] \textbf{ENEM 2010}
(Exame Nacional do Ensino Médio).
\emph{INEP-Instituto Nacional de Estudos e Pesquisas Educacionais Anísio Teixeira}.{Ministério da Educação}. 

\bibitem[ENEM(2025)]{ENEM(2025)}
[5] \textbf{ENEM 2025}
(Exame Nacional do Ensino Médio).
\emph{INEP-Instituto Nacional de Estudos e Pesquisas Educacionais Anísio Teixeira}.{Ministério da Educação}.

\bibitem[ENEM(2024)]{ENEM(2024)}
[6] \textbf{ENEM 2024}
(Exame Nacional do Ensino Médio).
\emph{INEP-Instituto Nacional de Estudos e Pesquisas Educacionais Anísio Teixeira}.{Ministério da Educação}.

\bibitem[ENEM(2022)]{ENEM(2022)}
[7] \textbf{ENEM 2022}
(Exame Nacional do Ensino Médio).
\emph{INEP-Instituto Nacional de Estudos e Pesquisas Educacionais Anísio Teixeira}.{Ministério da Educação}.

\bibitem[ENEM(2020)]{ENEM(2020)}
[8] \textbf{ENEM 2020}
(Exame Nacional do Ensino Médio).
\emph{INEP-Instituto Nacional de Estudos e Pesquisas Educacionais Anísio Teixeira}.{Ministério da Educação}.

\bibitem[ENEM(2017)]{ENEM(2017)}
[9] \textbf{ENEM 2017}
(Exame Nacional do Ensino Médio).
\emph{INEP-Instituto Nacional de Estudos e Pesquisas Educacionais Anísio Teixeira}.{Ministério da Educação}.

\bibitem[ENEM(2012)]{ENEM(2012)}
[10] \textbf{ENEM 2012}
(Exame Nacional do Ensino Médio).
\emph{INEP-Instituto Nacional de Estudos e Pesquisas Educacionais Anísio Teixeira}.{Ministério da Educação}.

\bibitem[ENEM(2013)]{ENEM(2013)}
[1] \textbf{ENEM 2013}
(Exame Nacional do Ensino Médio).
\emph{INEP-Instituto Nacional de Estudos e Pesquisas Educacionais Anísio Teixeira}.{Ministério da Educação}. 

\bibitem[ENEM(2009)]{ENEM(2009)}
[2] \textbf{ENEM 2009}
(Exame Nacional do Ensino Médio).
\emph{INEP-Instituto Nacional de Estudos e Pesquisas Educacionais Anísio Teixeira}.{Ministério da Educação}. 

\bibitem[ENEM(2011)]{ENEM(2011)}
[3] \textbf{ENEM 2011}
(Exame Nacional do Ensino Médio).
\emph{INEP-Instituto Nacional de Estudos e Pesquisas Educacionais Anísio Teixeira}.{Ministério da Educação}. 

\bibitem[ENEM(2010)]{ENEM(2010)}
[4] \textbf{ENEM 2010}
(Exame Nacional do Ensino Médio).
\emph{INEP-Instituto Nacional de Estudos e Pesquisas Educacionais Anísio Teixeira}.{Ministério da Educação}. 

\bibitem[ENEM(2025)]{ENEM(2025)}
[5] \textbf{ENEM 2025}
(Exame Nacional do Ensino Médio).
\emph{INEP-Instituto Nacional de Estudos e Pesquisas Educacionais Anísio Teixeira}.{Ministério da Educação}.

\bibitem[ENEM(2024)]{ENEM(2024)}
[6] \textbf{ENEM 2024}
(Exame Nacional do Ensino Médio).
\emph{INEP-Instituto Nacional de Estudos e Pesquisas Educacionais Anísio Teixeira}.{Ministério da Educação}.

\bibitem[ENEM(2022)]{ENEM(2022)}
[7] \textbf{ENEM 2022}
(Exame Nacional do Ensino Médio).
\emph{INEP-Instituto Nacional de Estudos e Pesquisas Educacionais Anísio Teixeira}.{Ministério da Educação}.

\bibitem[ENEM(2020)]{ENEM(2020)}
[8] \textbf{ENEM 2020}
(Exame Nacional do Ensino Médio).
\emph{INEP-Instituto Nacional de Estudos e Pesquisas Educacionais Anísio Teixeira}.{Ministério da Educação}.

\bibitem[ENEM(2017)]{ENEM(2017)}
[9] \textbf{ENEM 2017}
(Exame Nacional do Ensino Médio).
\emph{INEP-Instituto Nacional de Estudos e Pesquisas Educacionais Anísio Teixeira}.{Ministério da Educação}.

\bibitem[ENEM(2012)]{ENEM(2012)}
[10] \textbf{ENEM 2012}
(Exame Nacional do Ensino Médio).
\emph{INEP-Instituto Nacional de Estudos e Pesquisas Educacionais Anísio Teixeira}.{Ministério da Educação}.

\bibitem[ENEM(2014)]{ENEM(2014)}
[11] \textbf{ENEM 2014}
(Exame Nacional do Ensino Médio).
\emph{INEP-Instituto Nacional de Estudos e Pesquisas Educacionais Anísio Teixeira}.{Ministério da Educação}.

%===========================================================
\end{thebibliography}
\end{document}
