% LISTA DE EXERCÍCIOS Template using "book"
% Created by Milena Lima 
%Email:milenascimentolima@gmail.com
% Science Project at school 2017-FAPEAM
% View: https://www.overleaf.com/read/syjxcffdygch
%=======================================================
%------------LISTA DE EXERCÍCIOS
%=======================================================
\documentclass[12pt,a4paper,oneside,openany]{book} 
%=======================================================
%-------------PACOTES 
%=======================================================
\usepackage[top=1cm,left=1cm,right=1.5cm,bottom=2cm]{geometry}
\usepackage[T1]{fontenc}%Especif. a codif. de caracteres
\usepackage{ae}%Auxílio para fontes e pdf
\usepackage[utf8]{inputenc}
\usepackage{lipsum}%Gerar Texto Aleatório
\usepackage[brazil]{babel}%Traduzir para Português
\usepackage{indentfirst}%Faz indentações em parágrafos
\usepackage{graphicx}%Permite incluir figuras
\usepackage{subfig} %para criar sub figuras
\usepackage{float}% figuras
\usepackage{tabularx}
\usepackage{ragged2e}
\usepackage{multirow}
\usepackage[dvipsnames]{xcolor}%Admitir cores
\usepackage{amsmath,amssymb,amsthm}%Incluir expressões  matemáticas, equações, teoremas, símbolos, etc
\usepackage{lastpage}%Incluir Ficha catalográfica
\usepackage{epigraph}%Incluir Epígrafo
\usepackage{enumerate}
\usepackage{enumitem}
\newlist{questions}{enumerate}{3}
\setlist[questions]{label=\arabic*.}
\newcommand{\question}{\item}
\setlist[enumerate,1]{% (
leftmargin=*, itemsep=12pt, label={\textbf{\arabic*.)}}}
%---
\newlist{partes}{enumerate}{3}
\setlist[partes]{label=(\alph*)}
\newcommand{\parte}{\item}
%---
\newlist{subpartes}{enumerate}{3}
\setlist[subpartes]{label=\roman*)}
\newcommand{\subparte}{\item}
%---
\usepackage{array}
\usepackage{tikz}
\newcommand*\circled[1]{\tikz[baseline=(char.base)]{\node[shape=circle,draw,inner sep=2pt] (char) {#1};}}
\usepackage[sort&compress,round,comma,authoryear]{natbib}
\usepackage{makeidx}
\usepackage[colorlinks=true,urlcolor=magenta,citecolor=red,linkcolor=violet,bookmarks=true]{hyperref}
\usepackage{lscape}%Altera a orientação de uma página
\usepackage{pdflscape}
\usepackage{epstopdf} %converte figs eps em figs pdf
\usepackage{booktabs}
\usepackage{pdfpages}
\usepackage{textcomp}
\usepackage[many]{tcolorbox}
\usepackage{empheq}
\usepackage{tasks}%lista alfabética
\pagestyle{plain}
\usepackage{mhchem}
%================================================================
%------------DIGITE AQUI
%===============================================================
\newcommand{\orgao}{Universidade do Estado do Pará}
\newcommand{\instituto}{Movimento de Educação Popular e Inclusiva do Jurunas}
\newcommand{\departamento}{EEFM Arthur Porto}
\newcommand{\curso}{Ciências da Computação}
\newcommand{\professor}{Elias Sabát}
\newcommand{\disciplina}{Química}
\newcommand{\titulo}{4ª Lista de Exercícios: Funções Inorgânicas}
\newcommand{\data}{\today}
\newcommand{\aluno}{ALUNO:}
\newcommand{\email}{{\bf Professor: Isaias Tobelem}}
\newcommand{\turma}{kkkkkkkkkkkkkkkkkkkkkkkkkk}
%===========================================================
\newcommand{\X}{\mathbf{X}}
\newcommand{\x}{\mathbf{x}}
\newcommand{\Z}{\mathbf{Z}}
\newcommand{\z}{\mathbf{z}}
\newcommand{\y}{\boldsymbol{y}}
\newcommand{\balpha}{\mbox{${ \bm \alpha}$}}
\newcommand{\bmu}{\mbox{${\bm \mu}$}}
\newcommand{\bbeta}{\mbox{${\bm \beta}$}}
\newcommand{\bteta}{\mbox{${\bm \theta}$}}
\newcommand{\bgama}{\mbox{${\bm \gamma}$}}
\newcommand{\bxi}{\mbox{${\bm \xl}$}}
\newcommand{\bvarphi}{\mbox{${ \bm \varphi}$}}
\newcommand{\SZ}{\mbox{ $Z$}}
\newcommand{\muz}{\mu_{z,l}}
\newcommand{\muo}{\mu_{0,l}}
\newcommand{\etao}{\eta_{0,l}}
\newcommand{\etaz}{\eta_{z,l}}
\newcommand{\xbeta}{x_{l}\bgama}
\newcommand{\mui}{\mu_{l}}
\newcommand{\zetaind}{\zeta \mathtt{I}_{\{s_{l} \in Z \}}} 
\newcommand{\spz}{ s_{l} \in z}
\newcommand{\snpz}{ s_{l} \notin z}
\newcommand{\sps}{ s_{l} \in S}
\newcommand{\gphi}{ \Gamma(\phi)}
\newcommand{\scan}{ \Lambda_{z}}
\newcommand{\gmuop}{ \Gamma(\muo\phi)}
\newcommand{\gmuzp}{ \Gamma(\muz \phi)}
\newcommand{\dlobeta}{ \frac{\parteial l_{0}(\bgama, \phi, 0) }{\parteial \bgama}}
\newcommand{\lz}{  l_{z}(\bgama, \phi, \tau)}
\newcommand{\lo}{  l_{0}(\beta, \phi, 0)}
\newcommand{\E}{\mathbb{E}}
\newcommand{\dis}{\displaystyle}
\linespread{1.0}%espaço entre linhas
\begin{document}
%%%%%%%%%%%%%%%%%%%%%%%%%%%%%%%%%%%%%%%%%%%%%%%%%%%%%%%%
%                      CABEÇALHO                     %
%%%%%%%%%%%%%%%%%%%%%%%%%%%%%%%%%%%%%%%%%%%%%%%%%%%%%%%%
\begin{table}[H]
\centering
\begin{tabular*}{\textwidth}{l@{\extracolsep{\fill}}l@{\extracolsep{\fill}}}
\begin{tabular}[l]{@{}l@{}}\textbf{\orgao}\\\textbf{\instituto}\\\textbf{\departamento} \end{tabular} & \begin{tabular}[l]{@{}l@{}}\textbf{Professor: \professor}\\ {\email}\\ \textbf{Disciplina: \disciplina}\end{tabular}                                                       
\end{tabular*}
\end{table}
\begin{center}
\rule[2ex]{\textwidth}{1pt}\\
{\Large{\titulo}}
\end{center}
\rule[2ex]{\textwidth}{1pt}\\


\textbf{Assuntos:} Ácidos, Bases, Sais e Óxidos


\begin{questions}[label=\protect\circled{\bfseries\arabic*}]
%%%%%%%%%%%%%%%%%%%%%%%%%%%%%%%%%%%%%%%%%%%%%%%%%%%%%%%%
%                      Questões                   %
%%%%%%%%%%%%%%%%%%%%%%%%%%%%%%%%%%%%%%%%%%%%%%%%%%%%%%%%

%=========================================================
\question Os ácidos $HClO_{4}$, $H_2MnO_4$, $H_3PO_3$, $H_4Sb_2O_7$,
quanto ao número de hidrogênios ionizáveis, podem ser
classificados em:

\begin{partes}

 \parte monoácido, diácido, triácido, tetrácido.
\parte monoácido, diácido, triácido, triácido.
\parte monoácido, diácido, diácido, tetrácido
\parte monoácido, monoácido, diácido, triácido.

\end{partes}

\question Considere os ácidos oxigenados abaixo:

$$HNO_2 \quad HClO_3 \quad  H_2SO_3 \quad H_3PO_4$$

Seus nomes são, respectivamente:


\begin{partes}

\parte nitroso, clórico, sulfuroso, fosfórico.
\parte nítrico, clorídrico, sulfúrico, fosfórico.
\parte nítrico, hipocloroso, sulfuroso, fosforoso.
\parte nitroso, perclórico, sulfúrico, fosfórico.
\parte nítrico, cloroso, sulfídrico, hipofosforoso.

\end{partes}

\question  O ácido que é classificado como oxiácido,
diácido e é formado por átomos de três elementos químicos diferentes é:

\begin{partes}

\parte  $H_{2}S$
\parte HCN
\parte $HNO_{3}$
\parte $H_{4}P_{2}O_{7}$
\parte $H_{2}SO_{3}$

\end{partes}

\question Dê os nomes das seguintes bases:

\begin{partes}
 
 \parte $Mg(OH)_2$
\parte $Sn(OH)_2$
\parte Cs(OH)
\parte $Pt(OH)_4$
\parte $Hg(OH)_2$


\end{partes}

\question Urtiga é o nome genérico dado a diversas plantas da família das Urticáceas, cujas folhas são cobertas de
pêlos finos, os quais liberam ácido fórmico ($H_2CO_2$) que,
em contato com a pele, produz uma irritação.
Dos produtos de uso doméstico abaixo, o que você utilizaria para diminuir essa irritação é:


\begin{partes}
\parte  vinagre
\parte coalhada
\parte sal de cozinha
\parte leite de magnésia
\parte óleo
\end{partes}

\question Os derivados do potássio são amplamente utilizados na fabricação de explosivos, fogos de artifício, além
de outras aplicações. As fórmulas que correspondem ao
nitrato de potássio, perclorato de potássio, sulfato de
potássio e dicromato de potássio, são, respectivamente:

\begin{partes}
 
 \parte  $KNO_2$, $KClO_4$, $K_2SO_4$, $K_2Cr_2O_7$
\parte $KNO_3$, $KClO_4$, $K_2SO_4$, $K_2Cr_2O_7$
\parte $KNO_2$, $KClO_3$, $K_2SO_4$, $K_2Cr_2O_7$
\parte $KNO_2$, $KClO_4$, $K_2SO_4$, $K_2Cr_2O_4$
\parte $KNO_3$,, $KClO_3$, $K_2SO_4$, $K_2Cr_2O_7$
 
\end{partes}

\question  A água do mar pode ser fonte de sais usados na
fabricação de fermento em pó, de água sanitária e de
soro fisiológico. Os principais constituintes ativos desses
materiais são, respectivamente:

\begin{partes}
 
\parte $Na_2CO_3$, HCl e NaCl
\parte $NaHCO_3$, $Cl_2$ e $CaCl_2$
\parte $NaHCO_3$, NaOCl e NaCl
\parte $Na_2CO_3$, NaCl e KCl
\parte NaOCl, $NaHCO_3$ e NaCl
 
\end{partes}

\question O ferro é um dos elementos mais abundantes na crosta terrestre. Em Carajás, o principal minério de ferro é a hematita, substância constituída, principalmente, por óxido férrico (ou óxido de ferro III), cuja
fórmula é 

\begin{partes}
 \parte $FeO$
\parte $Fe_{3}O$
\parte $FeO_{3}$
\parte $Fe_{2}O_{3}$
\parte $Fe_{3}O_{2}$
\end{partes}

\question Sabe-se que a chuva ácida é formada pela dissolução, na água da chuva, de óxidos ácidos presentes
na atmosfera.
Entre os pares de óxidos relacionados, qual é constituído
apenas por óxidos que provocam a chuva ácida?

\begin{partes}

\parte $Na_{2}O$ e $NO_2$
\parte $CO_2$ e $SO_3$
\parte CO e NO
\parte $CO_2$ e MgO
\parte CO e $N_{2}O$

\end{partes}


\question No século XIX, o cientista Svante Arrhenius definiu ácidos como sendo as espécies químicas
que, ao se ionizarem em solução aquosa, liberam como cátion apenas o íon $H^+$. Considere as
seguintes substâncias, que apresentam hidrogênio em sua composição: $C_{2}H_6$, $H_{2}SO_{4}$, NaOH, $NH_{4}Cl$.
Dentre elas, aquela classificada como ácido, segundo a definição de Arrhenius, é:

\begin{partes}
 \parte $C_{2}H_6$

\parte $H_{2}SO_{4}$

\parte NaOH

\parte  $NH_{4}Cl$

\end{partes}

\question O oxigênio possui uma alta reatividade, podendo formar compostos com uma grande
variedade de elementos da Tabela Periódica. Dependendo do elemento, o óxido poderá ter
diferentes propriedades químicas.

Sobre os óxidos e as suas propriedades químicas, assinale a alternativa correta.

\begin{partes}


\parte O MgO reage com ácido e é considerado óxido ácido.

\parte Os óxidos dos ametais apresentam caráter iônico.

\parte A reação de um óxido básico com a água irá diminuir o pH da solução.

\parte O nitrogênio forma muitos óxidos. A partir do NO, pode ocorrer processo de redução,
formando $NO_{2}$.

\parte O $Al_{2}O_{3}$ tem caráter anfótero porque reage com ácido ou base.


\end{partes}


\question \citep{ENEM(2017)}
Realizou-se um experimento, utilizando-se o esquema mostrado na figura, para medir a
condutibilidade elétrica de soluções. Foram montados cinco kits contendo, cada um, três soluções
de mesma concentração, sendo uma de ácido, uma de base e outra de sal. Os kits analisados pelos
alunos foram:

\begin{figure}[H]
    \centering
    \includegraphics[width=0.5\linewidth]{1.png}
\end{figure}

Qual dos kits analisados provocou o acendimento da lâmpada com um brilho mais intenso nas três soluções?

\begin{partes}

\parte Kit 1.

\parte Kit 2.

\parte Kit 3.

 \parte Kit 4.

\parte Kit 5.

\end{partes}

\question \citep{ENEM(2015)}
Os calcários são materiais compostos por carbonato de cálcio, que podem atuar como
sorventes do dióxido de enxofre (SO2), um importante poluente atmosférico. As reações envolvidas
no processo são a ativação do calcário, por meio de calcinação, e a fixação do SO2 com a formação
de um sal de cálcio, como ilustrado pelas equações químicas simplificadas.
\begin{figure}[H]
    \centering
    \includegraphics[width=0.5\linewidth]{2.png}
\end{figure}

Considerando-se as reações envolvidas nesse processo de dessulfurização, a fórmula química
do sal de cálcio corresponde a:

\begin{partes}


\parte $CaSO_3$.

\parte $CaSO_4$.

\parte $CaS_{2}O_{8}$.

\parte $CaSO_{2}$.

\parte $CaS_{2}O_{7}$

\end{partes}


\question \citep{ENEM(2015)}
Em um experimento, colocou-se água até a metade da capacidade de um frasco de vidro e,
em seguida, adicionaram-se três gotas de solução alcoólica de fenolftaleína. Adicionou-se
bicarbonato de sódio comercial, em pequenas quantidades, até que a solução se tornasse rosa.
Dentro do frasco, acendeu-se um palito de fósforo, o qual foi apagado assim que a cabeça terminou de queimar. Imediatamente, o frasco foi tampado. Em seguida, agitou-se o frasco tampado e
observou-se o desaparecimento da cor rosa.

A explicação para o desaparecimento da cor rosa é que, com a combustão do palito de
fósforo, ocorreu o(a):

\begin{partes}

\parte formação de óxidos de caráter ácido.

\parte evaporação do indicador fenolftaleína.

\parte vaporização de parte da água do frasco.

\parte vaporização dos gases de caráter alcalino.

\parte aumento do pH da solução no interior do frasco

\end{partes}


\question \citep{ENEM(2009)}
O processo de industrialização tem gerado sérios problemas de ordem ambiental, econômica
e social, entre os quais se pode citar a chuva ácida. Os ácidos usualmente presentes em maiores
proporções na água da chuva são o H2CO3, formado pela reação do $CO_2$ atmosférico com a água, o
$HNO_3$, o $HNO_2$, o $H_{2}SO_{4}$ e o $H_{2}SO_{3}$. Esses quatro últimos são formados principalmente a partir da
reação da água com os óxidos de nitrogênio e de enxofre gerados pela queima de combustíveis
fósseis.
A formação de chuva mais ou menos ácida depende não só da concentração do ácido
formado, como também do tipo de ácido. Essa pode ser uma informação útil na elaboração de
estratégias para minimizar esse problema ambiental. Se consideradas concentrações idênticas, quais
dos ácidos citados no texto conferem maior acidez às águas das chuvas?


\begin{partes}

\parte $HNO_3$ e $HNO_2$.

\parte $H_{2}SO_{4}$ e $H_{2}SO_{3}$.

\parte $H_{2}SO_{3}$ e $HNO_2$.

\parte $H_{2}SO_{4}$ e $HNO_3$.

\parte $H_{2}CO_{3}$ e $H_{2}SO_{3}$.

\end{partes}

\question \citep{ENEM(2010)} 
As misturas efervescentes, em pó ou em comprimidos, são comuns para a administração de
vitamina C ou de medicamentos para azia. Essa forma farmacêutica sólida foi desenvolvida para
facilitar o transporte, aumentar a estabilidade de substâncias e, quando em solução, acelerar a
absorção do fármaco pelo organismo.

A matérias-primas que atuam na efervescência são, em geral, o ácido tartárico ou o ácido
cítrico que reagem com um sal de caráter básico, como o bicarbonato de sódio ($NaHCO_3$), quando
em contato com a água. A partir do contato da mistura efervescente com a água, ocorre uma série
de reações químicas simultâneas: liberação de íons, formação de ácido e liberação do gás carbônico
– gerando a efervescência.

As equações a seguir representam as etapas da reação da mistura efervescente na água, em
que foram omitidos os estados de agregação dos reagentes, e $H_{3}A$ representa o ácido cítrico.

I. $$NaHCO_{3}  \longrightarrow Na^{+} + HCO^{-3}$$

II. $$H_{2}CO_{3}  \longrightarrow H_{2}O + CO_{2}$$

III. $$HCO^{-3} + H^{+} \longrightarrow H_{2}CO_{3}$$

IV. $$H_{3}A \longrightarrow 3H^{+} + A^{-}$$

A ionização, a dissociação iônica, a formação do ácido e a liberação do gás ocorrem,
respectivamente, nas seguintes etapas:

\begin{partes}

\parte IV, I, II e III

\parte I, IV, III e II

\parte IV, III, I e II

\parte I, IV, II e III

\parte IV, I, III e II

\end{partes}



\question\citep{ENEM(2015)}
Em um experimento, colocou-se água até a metade da capacidade de um frasco de vidro e,
em seguida, adicionaram-se três gotas de solução alcoólica de fenolftaleína. Adicionou-se
bicarbonato de sódio comercial, em pequenas quantidades, até que a solução se tornasse rosa.
Dentro do frasco, acendeu-se um palito de fósforo, o qual foi apagado assim que a cabeça terminou de queimar. Imediatamente, o frasco foi tampado. Em seguida, agitou-se o frasco tampado e
observou-se o desaparecimento da cor rosa.

A explicação para o desaparecimento da cor rosa é que, com a combustão do palito de
fósforo, ocorreu o(a):

\begin{partes}

\parte formação de óxidos de caráter ácido.

\parte evaporação do indicador fenolftaleína.

\parte vaporização de parte da água do frasco.

\parte vaporização dos gases de caráter alcalino.

\parte aumento do pH da solução no interior do frasco

\end{partes}




%================================================================
\end{questions}
%===========================================================
%          BIBLIOGRAFIA
%===========================================================
\begin{thebibliography}{99}
\thispagestyle{empty}%myheadings
%===========================================================
%==============Livro
\bibitem[ENEM(2013)]{ENEM(2013)}
[1] \textbf{ENEM 2013}
(Exame Nacional do Ensino Médio).
\emph{INEP-Instituto Nacional de Estudos e Pesquisas Educacionais Anísio Teixeira}.{Ministério da Educação}. 

\bibitem[ENEM(2009)]{ENEM(2009)}
[2] \textbf{ENEM 2009}
(Exame Nacional do Ensino Médio).
\emph{INEP-Instituto Nacional de Estudos e Pesquisas Educacionais Anísio Teixeira}.{Ministério da Educação}. 

\bibitem[ENEM(2011)]{ENEM(2011)}
[3] \textbf{ENEM 2011}
(Exame Nacional do Ensino Médio).
\emph{INEP-Instituto Nacional de Estudos e Pesquisas Educacionais Anísio Teixeira}.{Ministério da Educação}. 

\bibitem[ENEM(2010)]{ENEM(2010)}
[4] \textbf{ENEM 2010}
(Exame Nacional do Ensino Médio).
\emph{INEP-Instituto Nacional de Estudos e Pesquisas Educacionais Anísio Teixeira}.{Ministério da Educação}. 

\bibitem[ENEM(2025)]{ENEM(2025)}
[5] \textbf{ENEM 2025}
(Exame Nacional do Ensino Médio).
\emph{INEP-Instituto Nacional de Estudos e Pesquisas Educacionais Anísio Teixeira}.{Ministério da Educação}.

\bibitem[ENEM(2024)]{ENEM(2024)}
[6] \textbf{ENEM 2024}
(Exame Nacional do Ensino Médio).
\emph{INEP-Instituto Nacional de Estudos e Pesquisas Educacionais Anísio Teixeira}.{Ministério da Educação}.

\bibitem[ENEM(2022)]{ENEM(2022)}
[7] \textbf{ENEM 2022}
(Exame Nacional do Ensino Médio).
\emph{INEP-Instituto Nacional de Estudos e Pesquisas Educacionais Anísio Teixeira}.{Ministério da Educação}.

\bibitem[ENEM(2020)]{ENEM(2020)}
[8] \textbf{ENEM 2020}
(Exame Nacional do Ensino Médio).
\emph{INEP-Instituto Nacional de Estudos e Pesquisas Educacionais Anísio Teixeira}.{Ministério da Educação}.

\bibitem[ENEM(2017)]{ENEM(2017)}
[9] \textbf{ENEM 2017}
(Exame Nacional do Ensino Médio).
\emph{INEP-Instituto Nacional de Estudos e Pesquisas Educacionais Anísio Teixeira}.{Ministério da Educação}.

\bibitem[ENEM(2012)]{ENEM(2012)}
[10] \textbf{ENEM 2012}
(Exame Nacional do Ensino Médio).
\emph{INEP-Instituto Nacional de Estudos e Pesquisas Educacionais Anísio Teixeira}.{Ministério da Educação}.

\bibitem[ENEM(2013)]{ENEM(2013)}
[1] \textbf{ENEM 2013}
(Exame Nacional do Ensino Médio).
\emph{INEP-Instituto Nacional de Estudos e Pesquisas Educacionais Anísio Teixeira}.{Ministério da Educação}. 

\bibitem[ENEM(2009)]{ENEM(2009)}
[2] \textbf{ENEM 2009}
(Exame Nacional do Ensino Médio).
\emph{INEP-Instituto Nacional de Estudos e Pesquisas Educacionais Anísio Teixeira}.{Ministério da Educação}. 

\bibitem[ENEM(2011)]{ENEM(2011)}
[3] \textbf{ENEM 2011}
(Exame Nacional do Ensino Médio).
\emph{INEP-Instituto Nacional de Estudos e Pesquisas Educacionais Anísio Teixeira}.{Ministério da Educação}. 

\bibitem[ENEM(2010)]{ENEM(2010)}
[4] \textbf{ENEM 2010}
(Exame Nacional do Ensino Médio).
\emph{INEP-Instituto Nacional de Estudos e Pesquisas Educacionais Anísio Teixeira}.{Ministério da Educação}. 

\bibitem[ENEM(2025)]{ENEM(2025)}
[5] \textbf{ENEM 2025}
(Exame Nacional do Ensino Médio).
\emph{INEP-Instituto Nacional de Estudos e Pesquisas Educacionais Anísio Teixeira}.{Ministério da Educação}.

\bibitem[ENEM(2024)]{ENEM(2024)}
[6] \textbf{ENEM 2024}
(Exame Nacional do Ensino Médio).
\emph{INEP-Instituto Nacional de Estudos e Pesquisas Educacionais Anísio Teixeira}.{Ministério da Educação}.

\bibitem[ENEM(2022)]{ENEM(2022)}
[7] \textbf{ENEM 2022}
(Exame Nacional do Ensino Médio).
\emph{INEP-Instituto Nacional de Estudos e Pesquisas Educacionais Anísio Teixeira}.{Ministério da Educação}.

\bibitem[ENEM(2020)]{ENEM(2020)}
[8] \textbf{ENEM 2020}
(Exame Nacional do Ensino Médio).
\emph{INEP-Instituto Nacional de Estudos e Pesquisas Educacionais Anísio Teixeira}.{Ministério da Educação}.

\bibitem[ENEM(2017)]{ENEM(2017)}
[9] \textbf{ENEM 2017}
(Exame Nacional do Ensino Médio).
\emph{INEP-Instituto Nacional de Estudos e Pesquisas Educacionais Anísio Teixeira}.{Ministério da Educação}.

\bibitem[ENEM(2012)]{ENEM(2012)}
[10] \textbf{ENEM 2012}
(Exame Nacional do Ensino Médio).
\emph{INEP-Instituto Nacional de Estudos e Pesquisas Educacionais Anísio Teixeira}.{Ministério da Educação}.

\bibitem[ENEM(2014)]{ENEM(2014)}
[11] \textbf{ENEM 2014}
(Exame Nacional do Ensino Médio).
\emph{INEP-Instituto Nacional de Estudos e Pesquisas Educacionais Anísio Teixeira}.{Ministério da Educação}.

\bibitem[ENEM(2015)]{ENEM(2015)}
[11] \textbf{ENEM 2015}
(Exame Nacional do Ensino Médio).
\emph{INEP-Instituto Nacional de Estudos e Pesquisas Educacionais Anísio Teixeira}.{Ministério da Educação}.

%===========================================================
\end{thebibliography}
\end{document}
