% LISTA DE EXERCÍCIOS Template using "book"
% Created by Milena Lima 
%Email:milenascimentolima@gmail.com
% Science Project at school 2017-FAPEAM
% View: https://www.overleaf.com/read/syjxcffdygch
%=======================================================
%------------LISTA DE EXERCÍCIOS
%=======================================================
\documentclass[12pt,a4paper,oneside,openany]{book} 
%=======================================================
%-------------PACOTES 
%=======================================================
\usepackage[top=1cm,left=1cm,right=1.5cm,bottom=2cm]{geometry}
\usepackage[T1]{fontenc}%Especif. a codif. de caracteres
\usepackage{ae}%Auxílio para fontes e pdf
\usepackage[utf8]{inputenc}
\usepackage{lipsum}%Gerar Texto Aleatório
\usepackage[brazil]{babel}%Traduzir para Português
\usepackage{indentfirst}%Faz indentações em parágrafos
\usepackage{graphicx}%Permite incluir figuras
\usepackage{subfig} %para criar sub figuras
\usepackage{float}% figuras
\usepackage{tabularx}
\usepackage{ragged2e}
\usepackage{multirow}
\usepackage[dvipsnames]{xcolor}%Admitir cores
\usepackage{amsmath,amssymb,amsthm}%Incluir expressões  matemáticas, equações, teoremas, símbolos, etc
\usepackage{lastpage}%Incluir Ficha catalográfica
\usepackage{epigraph}%Incluir Epígrafo
\usepackage{enumerate}
\usepackage{enumitem}
\newlist{questions}{enumerate}{3}
\setlist[questions]{label=\arabic*.}
\newcommand{\question}{\item}
\setlist[enumerate,1]{% (
leftmargin=*, itemsep=12pt, label={\textbf{\arabic*.)}}}
%---
\newlist{partes}{enumerate}{3}
\setlist[partes]{label=(\alph*)}
\newcommand{\parte}{\item}
%---
\newlist{subpartes}{enumerate}{3}
\setlist[subpartes]{label=\roman*)}
\newcommand{\subparte}{\item}
%---
\usepackage{array}
\usepackage{tikz}
\newcommand*\circled[1]{\tikz[baseline=(char.base)]{\node[shape=circle,draw,inner sep=2pt] (char) {#1};}}
\usepackage[sort&compress,round,comma,authoryear]{natbib}
\usepackage{makeidx}
\usepackage[colorlinks=true,urlcolor=magenta,citecolor=red,linkcolor=violet,bookmarks=true]{hyperref}
\usepackage{lscape}%Altera a orientação de uma página
\usepackage{pdflscape}
\usepackage{epstopdf} %converte figs eps em figs pdf
\usepackage{booktabs}
\usepackage{pdfpages}
\usepackage{textcomp}
\usepackage[many]{tcolorbox}
\usepackage{empheq}
\usepackage{tasks}%lista alfabética
\pagestyle{plain}
\usepackage{mhchem}
%================================================================
%------------DIGITE AQUI
%===============================================================
\newcommand{\orgao}{Universidade do Estado do Pará}
\newcommand{\instituto}{Movimento de Educação Popular e Inclusiva do Jurunas}
\newcommand{\departamento}{EEFM Arthur Porto}
\newcommand{\curso}{Ciências da Computação}
\newcommand{\professor}{Elias Sabát}
\newcommand{\disciplina}{Química}
\newcommand{\titulo}{9\textsuperscript{a} Lista de Exercícios: Introdução à Orgânica}
\newcommand{\data}{\today}
\newcommand{\aluno}{ALUNO:}
\newcommand{\email}{{\bf Professor: Isaias Tobelem}}
\newcommand{\turma}{kkkkkkkkkkkkkkkkkkkkkkkkkk}
%===========================================================
\newcommand{\X}{\mathbf{X}}
\newcommand{\x}{\mathbf{x}}
\newcommand{\Z}{\mathbf{Z}}
\newcommand{\z}{\mathbf{z}}
\newcommand{\y}{\boldsymbol{y}}
\newcommand{\balpha}{\mbox{${ \bm \alpha}$}}
\newcommand{\bmu}{\mbox{${\bm \mu}$}}
\newcommand{\bbeta}{\mbox{${\bm \beta}$}}
\newcommand{\bteta}{\mbox{${\bm \theta}$}}
\newcommand{\bgama}{\mbox{${\bm \gamma}$}}
\newcommand{\bxi}{\mbox{${\bm \xl}$}}
\newcommand{\bvarphi}{\mbox{${ \bm \varphi}$}}
\newcommand{\SZ}{\mbox{ $Z$}}
\newcommand{\muz}{\mu_{z,l}}
\newcommand{\muo}{\mu_{0,l}}
\newcommand{\etao}{\eta_{0,l}}
\newcommand{\etaz}{\eta_{z,l}}
\newcommand{\xbeta}{x_{l}\bgama}
\newcommand{\mui}{\mu_{l}}
\newcommand{\zetaind}{\zeta \mathtt{I}_{\{s_{l} \in Z \}}} 
\newcommand{\spz}{ s_{l} \in z}
\newcommand{\snpz}{ s_{l} \notin z}
\newcommand{\sps}{ s_{l} \in S}
\newcommand{\gphi}{ \Gamma(\phi)}
\newcommand{\scan}{ \Lambda_{z}}
\newcommand{\gmuop}{ \Gamma(\muo\phi)}
\newcommand{\gmuzp}{ \Gamma(\muz \phi)}
\newcommand{\dlobeta}{ \frac{\parteial l_{0}(\bgama, \phi, 0) }{\parteial \bgama}}
\newcommand{\lz}{  l_{z}(\bgama, \phi, \tau)}
\newcommand{\lo}{  l_{0}(\beta, \phi, 0)}
\newcommand{\E}{\mathbb{E}}
\newcommand{\dis}{\displaystyle}
\linespread{1.0}%espaço entre linhas
\begin{document}
%%%%%%%%%%%%%%%%%%%%%%%%%%%%%%%%%%%%%%%%%%%%%%%%%%%%%%%%
%                      CABEÇALHO                     %
%%%%%%%%%%%%%%%%%%%%%%%%%%%%%%%%%%%%%%%%%%%%%%%%%%%%%%%%
\begin{table}[H]
\centering
\begin{tabular*}{\textwidth}{l@{\extracolsep{\fill}}l@{\extracolsep{\fill}}}
\begin{tabular}[l]{@{}l@{}}\textbf{\orgao}\\\textbf{\instituto}\\\textbf{\departamento} \end{tabular} & \begin{tabular}[l]{@{}l@{}}\textbf{Professor: \professor}\\ {\email}\\ \textbf{Disciplina: \disciplina}\end{tabular}                                                       
\end{tabular*}
\end{table}
\begin{center}
\rule[2ex]{\textwidth}{1pt}\\
{\Large{\titulo}}
\end{center}
\rule[2ex]{\textwidth}{1pt}\\

\textbf{Assuntos:} Postulados de Kekulé, Hibridização do Carbono, Classificação do Carbono e das Cadeias Carbônicas e Hidrocarbonetos

\begin{questions}[label=\protect\circled{\bfseries\arabic*}]
%%%%%%%%%%%%%%%%%%%%%%%%%%%%%%%%%%%%%%%%%%%%%%%%%%%%%%%%
%                      Questões                   %
%%%%%%%%%%%%%%%%%%%%%%%%%%%%%%%%%%%%%%%%%%%%%%%%%%%%%%%%

%=========================================================

\question O ácido adípico,

\begin{figure}[H]
    \centering
    \includegraphics[width=0.5\linewidth]{1.png}
    
\end{figure}

matéria-prima para a produção de náilon, apresenta ca-
deia carbônica:

\begin{partes}
  \parte saturada, homogênea e ramificada.
  \parte saturada, heterogênea e normal.
\parte insaturada, homogênea e ramificada.
\parte saturada, homogênea e normal.
\parte insaturada, homogênea e normal.

\end{partes}

\question O linalol, substância isolada do óleo de alfazema,
apresenta a seguinte fórmula estrutural:

\begin{figure}[H]
    \centering
    \includegraphics[width=0.5\linewidth]{2.png}
    
\end{figure}

Essa cadeia carbônica é classificada como:

\begin{partes}
  \parte acíclica, normal, insaturada e homogênea.
\parte acíclica, ramificada, insaturada e homogênea.
\parte alicíclica, ramificada, insaturada e homogênea.
\parte alicíclica, normal, saturada e heterogênea.
\parte acíclica, ramificada, saturada e heterogênea.
\end{partes}

\question O citral, composto de fórmula

\begin{figure}[H]
    \centering
    \includegraphics[width=0.5\linewidth]{3.png}
    
\end{figure}

tem forte sabor de limão e é empregado em alimentos
para dar sabor e aroma cítricos. Sua cadeia carbônica é
classificada como:

\begin{partes}
  \parte homogênea, insaturada e ramificada.
\parte homogênea, saturada e normal.
\parte homogênea, insaturada e aromática.
\parte heterogênea, insaturada e ramificada.
\end{partes}


\question O mirceno, responsável pelo “gosto azedo da
cerveja”, é representado pela estrutura:

\begin{figure}[H]
    \centering
    \includegraphics[width=0.4\linewidth]{4.png}
    
\end{figure}

Considerando o composto indicado, identifique a alternativa correta quanto à classificação da cadeia.

\begin{partes}
  \parte acíclica, homogênea, saturada
\parte acíclica, heterogênea, insaturada
\parte cíclica, heterogênea, insaturada
\parte aberta, homogênea, saturada
\parte aberta, homogênea, insaturada
\end{partes}


\question Quando uma pessoa “leva um susto”, a suprarenal produz uma maior quantidade de adrenalina, que
é lançada na corrente sanguínea. Analisando a fórmula
estrutural da adrenalina,

\begin{figure}[H]
    \centering
    \includegraphics[width=0.5\linewidth]{5.png}
    
\end{figure}

podemos concluir que a cadeia orgânica ligada ao anel
aromático é:

\begin{partes}
  \parte aberta, saturada e homogênea.
\parte aberta, saturada e heterogênea.
\parte aberta, insaturada e heterogênea.
\parte fechada, insaturada e homogênea.
\parte fechada, insaturada e heterogênea.
\end{partes}

\question  O composto de fórmula

\begin{figure}[H]
    \centering
    \includegraphics[width=0.5\linewidth]{6.png}
    
\end{figure}

apresenta quantos carbonos primários, secundários,
terciários e quaternários, respectivamente?

\begin{partes}
  \parte  5, 5, 2 e 1
\parte 5, 4, 3 e 1
\parte 7, 4, 1 e 1
\parte 6, 4, 1 e 2
\parte 7, 3, 1 e 2
\end{partes}



%================================================================
\end{questions}
%===========================================================
%          BIBLIOGRAFIA
%===========================================================
\begin{thebibliography}{99}
\thispagestyle{empty}%myheadings
%===========================================================
%==============Livro
\bibitem[ENEM(2013)]{ENEM(2013)}
[1] \textbf{ENEM 2013}
(Exame Nacional do Ensino Médio).
\emph{INEP-Instituto Nacional de Estudos e Pesquisas Educacionais Anísio Teixeira}.{Ministério da Educação}. 

\bibitem[ENEM(2009)]{ENEM(2009)}
[2] \textbf{ENEM 2009}
(Exame Nacional do Ensino Médio).
\emph{INEP-Instituto Nacional de Estudos e Pesquisas Educacionais Anísio Teixeira}.{Ministério da Educação}. 

\bibitem[ENEM(2011)]{ENEM(2011)}
[3] \textbf{ENEM 2011}
(Exame Nacional do Ensino Médio).
\emph{INEP-Instituto Nacional de Estudos e Pesquisas Educacionais Anísio Teixeira}.{Ministério da Educação}. 

\bibitem[ENEM(2010)]{ENEM(2010)}
[4] \textbf{ENEM 2010}
(Exame Nacional do Ensino Médio).
\emph{INEP-Instituto Nacional de Estudos e Pesquisas Educacionais Anísio Teixeira}.{Ministério da Educação}. 

\bibitem[ENEM(2025)]{ENEM(2025)}
[5] \textbf{ENEM 2025}
(Exame Nacional do Ensino Médio).
\emph{INEP-Instituto Nacional de Estudos e Pesquisas Educacionais Anísio Teixeira}.{Ministério da Educação}.

\bibitem[ENEM(2024)]{ENEM(2024)}
[6] \textbf{ENEM 2024}
(Exame Nacional do Ensino Médio).
\emph{INEP-Instituto Nacional de Estudos e Pesquisas Educacionais Anísio Teixeira}.{Ministério da Educação}.

\bibitem[ENEM(2022)]{ENEM(2022)}
[7] \textbf{ENEM 2022}
(Exame Nacional do Ensino Médio).
\emph{INEP-Instituto Nacional de Estudos e Pesquisas Educacionais Anísio Teixeira}.{Ministério da Educação}.

\bibitem[ENEM(2020)]{ENEM(2020)}
[8] \textbf{ENEM 2020}
(Exame Nacional do Ensino Médio).
\emph{INEP-Instituto Nacional de Estudos e Pesquisas Educacionais Anísio Teixeira}.{Ministério da Educação}.

\bibitem[ENEM(2017)]{ENEM(2017)}
[9] \textbf{ENEM 2017}
(Exame Nacional do Ensino Médio).
\emph{INEP-Instituto Nacional de Estudos e Pesquisas Educacionais Anísio Teixeira}.{Ministério da Educação}.

\bibitem[ENEM(2012)]{ENEM(2012)}
[10] \textbf{ENEM 2012}
(Exame Nacional do Ensino Médio).
\emph{INEP-Instituto Nacional de Estudos e Pesquisas Educacionais Anísio Teixeira}.{Ministério da Educação}.

\bibitem[ENEM(2023)]{ENEM(2023)}
[1] \textbf{ENEM 2013}
(Exame Nacional do Ensino Médio).
\emph{INEP-Instituto Nacional de Estudos e Pesquisas Educacionais Anísio Teixeira}.{Ministério da Educação}. 

\bibitem[ENEM(2009)]{ENEM(2009)}
[2] \textbf{ENEM 2009}
(Exame Nacional do Ensino Médio).
\emph{INEP-Instituto Nacional de Estudos e Pesquisas Educacionais Anísio Teixeira}.{Ministério da Educação}. 

\bibitem[ENEM(2011)]{ENEM(2011)}
[3] \textbf{ENEM 2011}
(Exame Nacional do Ensino Médio).
\emph{INEP-Instituto Nacional de Estudos e Pesquisas Educacionais Anísio Teixeira}.{Ministério da Educação}. 

\bibitem[ENEM(2010)]{ENEM(2010)}
[4] \textbf{ENEM 2010}
(Exame Nacional do Ensino Médio).
\emph{INEP-Instituto Nacional de Estudos e Pesquisas Educacionais Anísio Teixeira}.{Ministério da Educação}. 

\bibitem[ENEM(2025)]{ENEM(2025)}
[5] \textbf{ENEM 2025}
(Exame Nacional do Ensino Médio).
\emph{INEP-Instituto Nacional de Estudos e Pesquisas Educacionais Anísio Teixeira}.{Ministério da Educação}.

\bibitem[ENEM(2024)]{ENEM(2024)}
[6] \textbf{ENEM 2024}
(Exame Nacional do Ensino Médio).
\emph{INEP-Instituto Nacional de Estudos e Pesquisas Educacionais Anísio Teixeira}.{Ministério da Educação}.

\bibitem[ENEM(2022)]{ENEM(2022)}
[7] \textbf{ENEM 2022}
(Exame Nacional do Ensino Médio).
\emph{INEP-Instituto Nacional de Estudos e Pesquisas Educacionais Anísio Teixeira}.{Ministério da Educação}.

\bibitem[ENEM(2020)]{ENEM(2020)}
[8] \textbf{ENEM 2020}
(Exame Nacional do Ensino Médio).
\emph{INEP-Instituto Nacional de Estudos e Pesquisas Educacionais Anísio Teixeira}.{Ministério da Educação}.

\bibitem[ENEM(2017)]{ENEM(2017)}
[9] \textbf{ENEM 2017}
(Exame Nacional do Ensino Médio).
\emph{INEP-Instituto Nacional de Estudos e Pesquisas Educacionais Anísio Teixeira}.{Ministério da Educação}.

\bibitem[ENEM(2012)]{ENEM(2012)}
[10] \textbf{ENEM 2012}
(Exame Nacional do Ensino Médio).
\emph{INEP-Instituto Nacional de Estudos e Pesquisas Educacionais Anísio Teixeira}.{Ministério da Educação}.

\bibitem[ENEM(2014)]{ENEM(2014)}
[11] \textbf{ENEM 2014}
(Exame Nacional do Ensino Médio).
\emph{INEP-Instituto Nacional de Estudos e Pesquisas Educacionais Anísio Teixeira}.{Ministério da Educação}.

\bibitem[ENEM(2015)]{ENEM(2015)}
[11] \textbf{ENEM 2015}
(Exame Nacional do Ensino Médio).
\emph{INEP-Instituto Nacional de Estudos e Pesquisas Educacionais Anísio Teixeira}.{Ministério da Educação}.

%===========================================================
\end{thebibliography}
\end{document}
